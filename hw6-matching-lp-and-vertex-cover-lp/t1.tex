%! Author = jinho
%! Date = 5/31/2020

\documentclass[UTF8, a4paper, linespread=1.5]{article}

\usepackage{tcolorbox, listings, algorithm, minted, algpseudocode}
\usepackage{geometry, savesym, amsmath, enumerate, indentfirst, color, amsthm, bm, extarrows, ulem}
\usepackage{amssymb}
\usepackage{nameref, hyperref}
 \geometry{top=3cm, bottom=3cm, left=1.5cm, right=1.5cm}

\usepackage{enumitem}
\setenumerate[1]{itemsep=0pt,partopsep=0pt,parsep=\parskip,topsep=5pt}
\setitemize[1]{itemsep=0pt,partopsep=0pt,parsep=\parskip,topsep=5pt}

\renewcommand\contentsname{Contents}

\tcbuselibrary{skins, breakable, theorems}

% \setlength{\leftskip}{10pt}
\setlength{\parindent}{10pt}
% \setlength{\parskip}{2em}
\renewcommand{\baselinestretch}{1.3}

\newcounter{RomanNumber}
\newcommand{\mrm}[1]{(\setcounter{RomanNumber}{#1}\Roman{RomanNumber})}

\newtcbtheorem{thm}{}
  {enhanced, theorem name and number, code={\edef\@currentlabelname{#2}}, 
  frame code={
        % \path[thick, draw] (frame.north west) -| (frame.north east) -| (frame.south east) -| (frame.south west) -| (frame.north west);
        \path[thick, draw] (frame.north west)  +(.5\baselineskip,0) -| +(0,-.5\baselineskip);
        % \path[thick, draw] (frame.north east) +(-.5\baselineskip,0) -| +(0,-.5\baselineskip);
        % \path[thick, draw] (frame.south west) +(.5\baselineskip,0) -| +(0,.5\baselineskip);
        \path[thick, draw] (frame.south east) +(-.5\baselineskip,0) -| +(0,.5\baselineskip);
    },
    left=1mm, right=1mm, top=1mm, bottom=1mm,
    colback=black!5,
    colframe=red!75!black,
    colbacktitle=black!0,
    coltitle=black!100,
    fonttitle=\bfseries}{thm}


\usepackage{environ}
\RenewEnviron{math}{%
\begin{align*}
\BODY
\end{align*}
}

\title{CS217 -- Algorithm Design and Analysis \\ Homework 5}
\date{\today}
\author{Not Strong Enough}


% Document
\begin{document}
    \begin{thm}{}{}
        Let $v(G)$ denote the size of a maximum matching of $G$.
        Obviously, $val(MLP(G))\ge v(G)$ for all graphs.
        Show that $v(G)=val(MLP(G))$ for all bipartite graphs $G$.
        Do this without referring to Konig's Theorem.
    \end{thm}
    \begin{proof}[Solution]
        Suppose $X$ is a solution to $MLP(G)$.
        $E_F$ is the set of edges with fractional values in X.
        First do as follows:
        \begin{itemize}
            \item[(1)] If $E_F$ doesn't contain a cycle, terminate.
            \item[(2)] Find a cycle in $E_F$ with fractional values, denote it $C:=\{e_i\in E_F :1\le i \le n,i\in\mathbb{N}\}$.
            \item[(3)] Add $x_{e_{2k-1}}$ by $\epsilon$ and $x_{e_{2k}}$ by $-\epsilon$ for $\{k\in \mathbb{N}:k<=n/2\}$.
            \item[(4)] Increase $\epsilon$ until there exists $i$ such that $x_{e_i}=0 \text{ or } 1$.
            \item[(5)] go to step(2) if there exists a cycle in $E_F$.
        \end{itemize}
        Since $G$ is a bipartite, C can only be a even cycle, so step (3) makes sense.
        If we modify the solution by step(3) , obviously the constraints of $MLP(G)$ will still be satisfied, and the target function will remain unchanged.
        The process will terminate because $|E_F|$ decreases by at least 1 in each iteration.

        Then do as follows:
        \begin{itemize}
            \item[(1)] If $E_F$ is empty, terminate.
            \item[(2)] Choose 2 vertex $v_1,v_2$ that $|\{x_e\in(0,1):v_1\in e\}|=1$, $|\{x_e\in(0,1):v_2\in e\}|=1$ and there's a path between $v_1$ and $v_2$ in $E_F$, denote it $P:=\{e_{i}'\in E_F:1\le i \le m,i\in\mathbb{N}\}$
            \item[(3)] Add $x_{e_{2k-1}'}$ by $\epsilon$ for $\{k\in\mathbb{N}:2k-1\le m\}$ and $x_{e_{2k}'}$ by $-\epsilon$ for $\{k\in\mathbb{N}:2k\le m\}$
            \item[(4)] Increase $\epsilon$ until there exists $i$ such that $x_{e_{i}'}=0 \text{ or } 1$
            \item[(5)] Go to step (2) if $E_F$ is not empty
        \end{itemize}
        Since there's no cycle in $E_F$, we can definitely choose $v_1,v_2$ that satisfy the requirements if $E_F$ is not empty.
        In each iteration, the target function will not decrease.
        Since $v_1,v_2$ can only be touched by edges with fractional value or value 0, the constraints of $v_1,v_2$ can be satisfied.
        In other words, $\sum_{e\in E:v_1\in e}x_e =x_{e_{1}'} \le 1$, $\sum_{e\in E:v_2\in e}x_e =x_{e_{m}'} \le 1$
        The constraints of other vertices $v$ in $P$ will obviously be maintained.
        The process will termintate because $|E_F|$ decreases by at least 1 in each iteration.

        Finally, after 2 processes, the solution becomes integral and the value of target function does not decrease, which means $\text{int-val}(MLP(G))\ge val(MLP(G))$.
        However we have  $\text{int-val}(MLP(G))\le val(MLP(G))$.
        So $v(G)=\text{int-val}(MLP(G))= val(MLP(G))$
    \end{proof}

\end{document}