\documentclass[UTF8, a4paper, linespread=1.5]{article}

\usepackage{tcolorbox, listings}
\usepackage{geometry, amsmath, enumerate, indentfirst, color, amsthm, bm, extarrows, ulem}
\usepackage{amssymb}
\usepackage{nameref, hyperref}
\usepackage{algorithm}
\usepackage{algorithmic}
 % \geometry{top=3cm, bottom=3cm, left=1.5cm, right=1.5cm}

\usepackage{enumitem}
\setenumerate[1]{itemsep=0pt,partopsep=0pt,parsep=\parskip,topsep=5pt}
\setitemize[1]{itemsep=0pt,partopsep=0pt,parsep=\parskip,topsep=5pt}

% \usepackage{adjustbox}

\renewcommand\contentsname{Contents}

\tcbuselibrary{skins, breakable, theorems}

% \setlength{\leftskip}{10pt}
\setlength{\parindent}{10pt}
% \setlength{\parskip}{2em}
\renewcommand{\baselinestretch}{1.3}

\newcounter{RomanNumber}
\newcommand{\mrm}[1]{(\setcounter{RomanNumber}{#1}\Roman{RomanNumber})}

\newtcbtheorem[number within=subsection]{thm}{}
  {enhanced, theorem name and number, code={\edef\@currentlabelname{#2}}, 
  frame code={
        % \path[thick, draw] (frame.north west) -| (frame.north east) -| (frame.south east) -| (frame.south west) -| (frame.north west);
        \path[thick, draw] (frame.north west)  +(.5\baselineskip,0) -| +(0,-.5\baselineskip);
        % \path[thick, draw] (frame.north east) +(-.5\baselineskip,0) -| +(0,-.5\baselineskip);
        % \path[thick, draw] (frame.south west) +(.5\baselineskip,0) -| +(0,.5\baselineskip);
        \path[thick, draw] (frame.south east) +(-.5\baselineskip,0) -| +(0,.5\baselineskip);
    },
    left=1mm, right=1mm, top=1mm, bottom=1mm,
    colback=black!5,
    colframe=red!75!black,
    colbacktitle=black!0,
    coltitle=black!100,
    fonttitle=\bfseries}{thm}


\usepackage{xparse}
\NewDocumentEnvironment{qte}{m}{\begin{tcolorbox}[breakable, leftrule=2mm, rightrule=-0.1mm, toprule=-0.1mm, bottomrule=-0.1mm, arc=0mm, colframe=black!30!white, colback=white, coltext=white!50!black]}{\\\rightline{#1}\end{tcolorbox}}

\title{Algorithm Analysis}
\date{\today}
\author{Not Strong Enough}

\begin{document}
\maketitle

\section{Bit Complexity of Euclid Algorithm}

\subsection{Exercise 1}

{\bfseries prove the more precise bound of the school method}

Assume $a$ has $n$ bits and $b$ has $k$ bits($k < n$), in the school method of division, 
calculate in rounds. 

In each round, extend $b$ by filling zeros into lower bits and make $b$ have same bit size as $a$. 
If the extended $b$ is smaller than $a$, let $a$ minus $b$. 

It is obvious that the first bit of $b$ is 1, so each minus will decrease the bit size of $a$. 
Besides, since the minus is made only when the extended $b$ is smaller than the current value of $a$, 
if the bit size of $a$ is smaller than $k$, the calculation will end. 
So there are at most $n-k+1$ rounds. 

Now consider the minus. Since the lower bits of extended $b$ are 0, only the first $k$ bits of extended $b$ 
costs. So each minus has $k$ operations. 

In all, there are at most $(n-k+1)*k$ operations, so the complexity is $O(k(n-k+1))=O(k(n-k))$ when $n\neq k$. 
(but if $n=k$, $O(k(n-k))=O(0)$, but sometimes the method has some operations: e.g., $a=11$ and $b=10$, 
there are 2 operations, which is not $O(0)$. )

\subsection{Exercise 2}

{\bfseries prove the complexity of Euclid algorithm is $O(n^2)$}

In each round, the Euclid algorithm calculates $a\%b$(assume $a\geq b$) which is less than $b$. 
If the result is not 0, it uses the result with b to do the new round. So we can assume that, in each round, 
the two calculated numbers are: 
$$(x_0, x_1), (x_1, x_2)\dots (x_{m-1}, x_m)$$
Where $x_0>x_1>\dots x_m, x_{m-1}\%x_m=0$. 
Let the bit size of $x_i$ be $t_i$, then $t_0=n, t_1=k$, 
using the result of Exercise 1 we can find that the total operation number is: 
\begin{align}
O(t_1(t_0-t_1))+O(t_2(t_1-t_2))+\dots + O(t_m(t_{m-1}-t_m)) &=O(\sum_{i=1}^m t_i t_{i-1}-\sum_{i=1}^m t_i^2)\\
&<O(\sum_{i=0}^{m-1} t_i^2 -\sum_{i=1}^m t_i^2)\\
&=O(t_0^2-t_m^2)=O(n^2)
\end{align}
So we can see the operation number of Euclid algorithm is $O(n^2)$
\end{document}
