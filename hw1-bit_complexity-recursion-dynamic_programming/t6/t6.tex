\documentclass[UTF8, a4paper, linespread=1.5]{article}

\usepackage{tcolorbox, listings}
\usepackage{geometry, savesym, amsmath, enumerate, indentfirst, color, amsthm, bm, extarrows, ulem}
\usepackage{amssymb}
\usepackage{nameref, hyperref}
 % \geometry{top=3cm, bottom=3cm, left=1.5cm, right=1.5cm}

\usepackage{enumitem}
\setenumerate[1]{itemsep=0pt,partopsep=0pt,parsep=\parskip,topsep=5pt}
\setitemize[1]{itemsep=0pt,partopsep=0pt,parsep=\parskip,topsep=5pt}

% \usepackage{adjustbox}

\renewcommand\contentsname{Contents}

\tcbuselibrary{skins, breakable, theorems}

% \setlength{\leftskip}{10pt}
\setlength{\parindent}{10pt}
% \setlength{\parskip}{2em}
\renewcommand{\baselinestretch}{1.3}

\newcounter{RomanNumber}
\newcommand{\mrm}[1]{(\setcounter{RomanNumber}{#1}\Roman{RomanNumber})}

\newtcbtheorem[number within=subsection]{thm}{}
  {enhanced, theorem name and number, code={\edef\@currentlabelname{#2}}, 
  frame code={
        % \path[thick, draw] (frame.north west) -| (frame.north east) -| (frame.south east) -| (frame.south west) -| (frame.north west);
        \path[thick, draw] (frame.north west)  +(.5\baselineskip,0) -| +(0,-.5\baselineskip);
        % \path[thick, draw] (frame.north east) +(-.5\baselineskip,0) -| +(0,-.5\baselineskip);
        % \path[thick, draw] (frame.south west) +(.5\baselineskip,0) -| +(0,.5\baselineskip);
        \path[thick, draw] (frame.south east) +(-.5\baselineskip,0) -| +(0,.5\baselineskip);
    },
    left=1mm, right=1mm, top=1mm, bottom=1mm,
    colback=black!5,
    colframe=red!75!black,
    colbacktitle=black!0,
    coltitle=black!100,
    fonttitle=\bfseries}{thm}


\usepackage{xparse}
\NewDocumentEnvironment{qte}{m}{\begin{tcolorbox}[breakable, leftrule=2mm, rightrule=-0.1mm, toprule=-0.1mm, bottomrule=-0.1mm, arc=0mm, colframe=black!30!white, colback=white, coltext=white!50!black]}{\\\rightline{#1}\end{tcolorbox}}

\title{Algorithm Design and Analysis}
\date{\today}
\author{Not Strong Enough}

\begin{document}
\maketitle

\section{Bit Complexity of Euclid Algorithm}

\begin{thm}{Exersise 6.}{}
    Remember the ``period'' algorithm for computing $F'_n := (F_n \mod k)$ discussed in class: (1) find some $i,j$ between $0$ and $k^2$ for which $F'_{i} =  F'_{j}$ and $F'_{i+1} = F'_{j+1}$. Then for $d := j-i$ the sequence $F'_{n}$ will repeat every $d$ steps, as there will be a cycle. This cycle can either be a ``true cycle'' or a ``lasso''. Show that a lasso cannot happen. That is, show that the smallest $i$ for which this happens is $0$.
\end{thm}

\begin{proof}
    We prove it by contradiction.
    
    Let $i_0 > 0$ be the smallest $i$ for which this happen, i.e, for some $j > 0$ we have: for $\forall p \geqslant i_0 > 0, F'_{p + j} = F'_p$. Since for $p \geqslant 1$, the fibonacci numbers are defined as $F_{p + 1} = F_{p} + F_{p - 1}$, we have $F'_{p + 1} = (F'_p + F'_{p - 1}) \mod k$. It follows that $F'_{p - 1} = (F'_{p + 1} - F'_p + k) \mod k$.
    
    Because $F'_{i_0} = F'_{i_0 + j}, F'_{i_0 + 1} = F'_{i_0 + 1 + j}$, we have
    \begin{align*}
        F'_{i_0 - 1} &= (F'_{i_0 + 1} - F'_{i_0} + k) \mod k \\
        &= (F'_{i_0 + 1 + j} - F'_{i_0 + j} + k) \mod k \\
        &= F'_{i_0 - 1 + j}
    \end{align*}
    , which is contradict with out hypothesis.
    
    Thus, $i_0$ isn't the smallest $i$, and we can conclude that the smallest $i$ is $0$.
\end{proof}

\end{document}

