\documentclass[UTF8, a4paper, linespread=1.5]{article}

\usepackage{tcolorbox, listings}
\usepackage{geometry, savesym, amsmath, enumerate, indentfirst, color, amsthm, bm, extarrows, ulem}
\usepackage{amssymb}
\usepackage{nameref, hyperref}
 % \geometry{top=3cm, bottom=3cm, left=1.5cm, right=1.5cm}

\usepackage{enumitem}
\setenumerate[1]{itemsep=0pt,partopsep=0pt,parsep=\parskip,topsep=5pt}
\setitemize[1]{itemsep=0pt,partopsep=0pt,parsep=\parskip,topsep=5pt}

% \usepackage{adjustbox}

\renewcommand\contentsname{Contents}

\tcbuselibrary{skins, breakable, theorems}

% \setlength{\leftskip}{10pt}
\setlength{\parindent}{10pt}
% \setlength{\parskip}{2em}
\renewcommand{\baselinestretch}{1.3}

\newcounter{RomanNumber}
\newcommand{\mrm}[1]{(\setcounter{RomanNumber}{#1}\Roman{RomanNumber})}

\newtcbtheorem[number within=subsection]{thm}{}
  {enhanced, theorem name and number, code={\edef\@currentlabelname{#2}}, 
  frame code={
        % \path[thick, draw] (frame.north west) -| (frame.north east) -| (frame.south east) -| (frame.south west) -| (frame.north west);
        \path[thick, draw] (frame.north west)  +(.5\baselineskip,0) -| +(0,-.5\baselineskip);
        % \path[thick, draw] (frame.north east) +(-.5\baselineskip,0) -| +(0,-.5\baselineskip);
        % \path[thick, draw] (frame.south west) +(.5\baselineskip,0) -| +(0,.5\baselineskip);
        \path[thick, draw] (frame.south east) +(-.5\baselineskip,0) -| +(0,.5\baselineskip);
    },
    left=1mm, right=1mm, top=1mm, bottom=1mm,
    colback=black!5,
    colframe=red!75!black,
    colbacktitle=black!0,
    coltitle=black!100,
    fonttitle=\bfseries}{thm}


\usepackage{xparse}
\NewDocumentEnvironment{qte}{m}{\begin{tcolorbox}[breakable, leftrule=2mm, rightrule=-0.1mm, toprule=-0.1mm, bottomrule=-0.1mm, arc=0mm, colframe=black!30!white, colback=white, coltext=white!50!black]}{\\\rightline{#1}\end{tcolorbox}}

\title{Combinatorics and Graph Theory}
\date{\today}
\author{Not Strong Enough}

\begin{document}
\maketitle

\section{Set Theory}

\begin{thm}{Definition}{}
    \begin{itemize}
        \item $[n] = \{1, \cdots , n\} $
        \item $A \subseteq [n]$, then $a$, the eigenvector of $A$, $\in \mathbb{F}_2^n$
        \item $A \times B := \{(a, b): a \in A, b \in B\} $, actually the ordered pair $(a, b)$ can be represented by $\{a, \{a, b\} \} $ without having to introduce new symbols.
    \end{itemize}
\end{thm}

\begin{itemize}
    \item 
        e.g. $A = \{1, 3, 4\} \subseteq [n]$, $a = (1, 0, 1, 1, 1)$.

        $A \triangle B \leftrightarrow a + b$, $+$ is under $\mathbb{F}_2^n$.
\end{itemize}

\begin{thm}{Cantor-Bernstein-Schröder Theorem}{}
    For set $A, B$ if there exists injections $f: A\to B$ and $g: B\to A$, then there is a bijection $h: A\to B$.
\end{thm}

\begin{itemize}
    \item 
        With this theorem, we can show that $\mathbb{N} ^k \sim  \mathbb{N} $ by construction two injections.

        $f_1: \mathbb{N} ^k \to \mathbb{N}, (a_1, a_2,\cdots ,a_n) \mapsto \prod_{i=1}^k p_i^{a_i} $, where $p_i$ are different prime numbers.

        $f_2: \mathbb{N} ^k \to  \mathbb{N} , (a_1, a_2, \cdots , a_n) \mapsto "a_1, a_2,\cdots ,a_n"$ in $13$ base with $',' = 11$.

    \item
        Furthermore, 
        \begin{thm}{}{}
            If any element of set $A$ can be represented by a \textbf{finite} string in a \textbf{countable} alphabet $\Sigma$, then $A$ is \textbf{countable} .
        \end{thm}

    \item
        And we can only put at most countable disjoint "plate"s in $\mathbb{R} ^2$. But there can be uncountable "ring"s.

        What about "8"s, that are two closed curve with exactly one shared point? And "y"s, that are three curves with shared starting point?
\end{itemize}

\begin{thm}{Definition}{}
    Let $A, B \subseteq U$, $A + B := \{a +^U b : a \in A, b \in B\} $.
\end{thm}

\begin{itemize}
    \item $U = \mathbb{R} $, then $|A + B| \ge s+t-1$

        Under which $U$ and $+^U$ can this hold? $\mathbb{Z} _5$ is a counter-example.
    \item ~ 
        \begin{thm}{The Cauchy-Davenport Theorem}{}
            $A, B \subseteq \mathbb{Z} _p, |A+B| \ge \min(p, s+t-1)$
        \end{thm}

        Induction on $|B|$.
\end{itemize}

\begin{thm}{Definition}{}
    $A^B := \{f: B \to A\} $
\end{thm}

\begin{itemize}
    \item 
        $f: B \to A$ $\leftrightarrow$ $A^{|B|}$

        e.g. $[10]^{[3]} \leftrightarrow [10]^3$, 
        $f \in [10]^{[3]} \leftrightarrow (f(1), f(2), f(3)) \in [10]^3$.

        \begin{itemize}
            \item 
                $u + v \leftrightarrow f + g$

            \item 
                (Cauchy-Schwarz)
                $\left<i, v \right>^2 \le  \left<u, u \right> \left<v, v \right> \leftrightarrow (\int fg \mathrm{d} x)^2 \le \int f^2 \mathrm{d} x + \int g^2 \mathrm{d} x$
        \end{itemize}
    \item 
        $f \in A^B$ devides $B$ into $|A|$ parts.
\end{itemize}

\begin{thm}{Definition}{}
    $A$ is a set, $r \in \mathbb{Z} $, $\displaystyle \binom A r := \{B \in A:  |B| = r\} \subseteq 2^A$ 
\end{thm}

\begin{itemize}
    \item e.g. $\displaystyle \binom A 0 = \{\varnothing\}, \binom A {-1} = \varnothing$.
    \item
        $n \in \mathbb{N}$, $r\in \mathbb{Z}$, $\displaystyle \binom n r := |\binom {[n]} r|$

        $\displaystyle \binom n r = \frac{(n)_r}{r!} = \frac{n!}{r!(n-r)!}$, should $r \ge 0$ and $r \le n$?
\end{itemize}

\end{document}

