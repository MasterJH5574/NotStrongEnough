\documentclass[UTF8, a4paper, linespread=1.5]{article}

\usepackage{tcolorbox, listings, algorithm, minted, algorithmic}
\usepackage{geometry, savesym, amsmath, enumerate, indentfirst, color, amsthm, bm, extarrows, ulem}
\usepackage{amssymb}
\usepackage{nameref, hyperref}
 % \geometry{top=3cm, bottom=3cm, left=1.5cm, right=1.5cm}

\usepackage{enumitem}
\setenumerate[1]{itemsep=0pt,partopsep=0pt,parsep=\parskip,topsep=5pt}
\setitemize[1]{itemsep=0pt,partopsep=0pt,parsep=\parskip,topsep=5pt}

% \usepackage{adjustbox}

\renewcommand\contentsname{Contents}

\tcbuselibrary{skins, breakable, theorems}

% \setlength{\leftskip}{10pt}
\setlength{\parindent}{10pt}
% \setlength{\parskip}{2em}
\renewcommand{\baselinestretch}{1.3}

\newcounter{RomanNumber}
\newcommand{\mrm}[1]{(\setcounter{RomanNumber}{#1}\Roman{RomanNumber})}

\newtcbtheorem{thm}{}
  {enhanced, theorem name and number, code={\edef\@currentlabelname{#2}}, 
  frame code={
        % \path[thick, draw] (frame.north west) -| (frame.north east) -| (frame.south east) -| (frame.south west) -| (frame.north west);
        \path[thick, draw] (frame.north west)  +(.5\baselineskip,0) -| +(0,-.5\baselineskip);
        % \path[thick, draw] (frame.north east) +(-.5\baselineskip,0) -| +(0,-.5\baselineskip);
        % \path[thick, draw] (frame.south west) +(.5\baselineskip,0) -| +(0,.5\baselineskip);
        \path[thick, draw] (frame.south east) +(-.5\baselineskip,0) -| +(0,.5\baselineskip);
    },
    left=1mm, right=1mm, top=1mm, bottom=1mm,
    colback=black!5,
    colframe=red!75!black,
    colbacktitle=black!0,
    coltitle=black!100,
    fonttitle=\bfseries}{thm}


\usepackage{xparse}
\NewDocumentEnvironment{qte}{m}{\begin{tcolorbox}[breakable, leftrule=2mm, rightrule=-0.1mm, toprule=-0.1mm, bottomrule=-0.1mm, arc=0mm, colframe=black!30!white, colback=white, coltext=white!50!black]}{\\\rightline{#1}\end{tcolorbox}}

\title{Combinatorics and Graph Theory}
\date{\today}
\author{Not Strong Enough}

\begin{document}
\maketitle

\begin{thm}{A Recursive Algorithm for the Binomial Coefficient}{}
   Using pseudocode, write a recursive algorithm computing
  ${n \choose k}$. Implement it in python! What is 
  the running time of your algorithm, in terms of $n$ and $k$? Would you say it is an efficient
  algorithm? Why or why not?
\end{thm}

\begin{algorithm}
    \caption{Binom\_BF: A Recursive Algorithm for the Binomial Coefficient.}
	\begin{algorithmic}
        \IF{$k=0$ or $k=n$}
        \RETURN $1$
        \ELSE
        \RETURN $Binom\_BF(n-1, k-1) + Binom\_BF(n-1, k)$
        \ENDIF
	\end{algorithmic}
\end{algorithm}

\begin{minted}{Python}
def binom_BF(n, k):
    if k==0 or k==n:
        return 1
    else:
        return binom_BF(n-1, k-1) + binom_BF(n-1, k)
\end{minted}

In order to count the number of operations in the recursive algorithm to calculate $\displaystyle \binom n k$. We shall take a look at how the values are calculated. While recursively calculating Fibonacci number $F_n$ can be represented by a tree, we can show this process by a grid graph. A value $\displaystyle \binom n k$, in grid $(n, k)$, is obtained by $(n-1, k-1)$ and $(n-1, k)$. This indicates that there are edges from $(n-1, k-1)$ and $(n-1, k)$ to $(n, k)$.

The recursive algorithm actually tries all possible paths to $(n, k)$, starting at $(i, j)$ where $j = 0$ or $j = i$. That is to say, the time that value is merged as $\displaystyle \binom i j = \binom {i-1}{j-1} + \binom {i-1} j$, is the number of paths going through $(i, j)$, which, by the definition of binomial coefficients, is $\displaystyle \binom {n-i}{k-j}$.

Considering the complexity of integer addition $O(\log \text{ bitsize})$, we can get the running time of the recursive algorithm: $$\sum_{i=0}^n\sum_{j=1}^{i-1} \binom {n-i}{k-j} \log  \binom i j < \sum_{i=0}^n\sum_{j=1}^{i-1} \binom {n-i}{k-j} \log  \binom n {\left\lfloor \frac{n}{2} \right\rfloor }.$$

Using the Stirling Formula, we have
$$O( \log \binom i j) = O((i+\frac{1}{2})\log i-(i-j+\frac{1}{2})\log (i-j)-(j+\frac{1}{2})\log j) = O(i).$$

The upper bound is
$$O(\sum_{i=0}^n\sum_{j=1}^{i-1} \binom {n-i}{k-j} \log  \binom n {\left\lfloor \frac{n}{2} \right\rfloor }) = O(n \binom n k).$$

And the lower bound is
$$\Omega(\binom nk).$$

\end{document}

