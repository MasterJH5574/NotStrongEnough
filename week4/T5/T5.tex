\documentclass[UTF8, a4paper, linespread=1.5]{article}

\usepackage{tcolorbox, listings, algorithm, minted, algpseudocode}
\usepackage{geometry, savesym, amsmath, enumerate, indentfirst, color, amsthm, bm, extarrows, ulem}
\usepackage{amssymb}
\usepackage{nameref, hyperref}
 \geometry{top=3cm, bottom=3cm, left=1.5cm, right=1.5cm}

\usepackage{enumitem}
\setenumerate[1]{itemsep=0pt,partopsep=0pt,parsep=\parskip,topsep=5pt}
\setitemize[1]{itemsep=0pt,partopsep=0pt,parsep=\parskip,topsep=5pt}

\renewcommand\contentsname{Contents}

\tcbuselibrary{skins, breakable, theorems}

% \setlength{\leftskip}{10pt}
\setlength{\parindent}{10pt}
% \setlength{\parskip}{2em}
\renewcommand{\baselinestretch}{1.3}

\newcounter{RomanNumber}
\newcommand{\mrm}[1]{(\setcounter{RomanNumber}{#1}\Roman{RomanNumber})}

\newtcbtheorem{thm}{}
  {enhanced, theorem name and number, code={\edef\@currentlabelname{#2}}, 
  frame code={
        % \path[thick, draw] (frame.north west) -| (frame.north east) -| (frame.south east) -| (frame.south west) -| (frame.north west);
        \path[thick, draw] (frame.north west)  +(.5\baselineskip,0) -| +(0,-.5\baselineskip);
        % \path[thick, draw] (frame.north east) +(-.5\baselineskip,0) -| +(0,-.5\baselineskip);
        % \path[thick, draw] (frame.south west) +(.5\baselineskip,0) -| +(0,.5\baselineskip);
        \path[thick, draw] (frame.south east) +(-.5\baselineskip,0) -| +(0,.5\baselineskip);
    },
    left=1mm, right=1mm, top=1mm, bottom=1mm,
    colback=black!5,
    colframe=red!75!black,
    colbacktitle=black!0,
    coltitle=black!100,
    fonttitle=\bfseries}{thm}


\usepackage{environ}
\RenewEnviron{math}{%
\begin{align*}
\BODY
\end{align*}
}

\title{CS217 -- Algorithm Design and Analysis \\ Homework 5}
\date{\today}
\author{Not Strong Enough}



\begin{document}

\begin{thm}{}{}
    Suppose you have a polynomial-time algorithm that, given a multigraph $H$, computes the number of spanning trees of $H$. Using this algorithm as a subroutine, design a polynomial-time algorithm that, given a weighted graph $G$, computes the number of minimum spanning trees of $G$.
\end{thm}

\begin{algorithm}
    \caption{Compute the number of minimun spanning tree of $G$}
    \begin{algorithmic}
        \Function{NumberOfSpanningTreesOfMultigraph}{$V$, $E$}

            some black magic...

        \EndFunction

        \Function{NumberOfMST}{$V$, $E$, $w$}

            \State $W \gets \{w(e_i): e_i \in E\}$
            \State $X \gets \varnothing$
            \State $Answer \gets 1$

            \For{$w_i \in W$ in increasing order}

                \State $E_{w_i} \gets \{e_i: w(e_i) = w_i\} $
                \State $V' \gets$ connected components of $G' = (V, X)$
                \State $E' \gets \{(A, B): (x, y) \in E, x \text{ is in component } A, y \text{ is in component } B\}$
                \State $Answer \gets Answer \times$ \Call{NumberOfSpanningTreesOfMultigraph}{$V'$, $E'$}
                \State $X \gets X \cup E_{w_i}$

            \EndFor

            \Return $Answer$

        \EndFunction
    \end{algorithmic}

\end{algorithm}

    The algorithm above can compute the number of MST of $G$ in polynomial time if we can get the number of spanning trees of multigraphs in polynomial time.
    \begin{itemize}
        \item Effectiveness(polynomial-time):

            It is polynomial-time because the for-loop will be executed $O(n)$ times, and every statement in the loop can be done in polynomial time.

        \item Correctness:

            Consider the process of Kruskal's Algorithm to find an MST, [problem 1] shows that $T_{w_i}$ will always have the same connected components. Thus, by adding edges of same weights into that graph, the resulting connected components will always be the same. This observation indicates that there will be multiple MSTs if and only if we can add $E_{w_i}$ into the graph in different ways, and for every $w_i$, the influence on answer is independent.

            Thus at each time we add edges of same weights into the graph together, and find the number of ways these edges can connect the current graph. The number of MSTs is the product of numbers of ways for every $w_i$.


    \end{itemize}
    
\end{document}
