	\documentclass[UTF8, a4paper, linespread=1.5]{article}

\usepackage{tcolorbox, listings}
\usepackage{geometry, savesym, amsmath, enumerate, indentfirst, color, amsthm, bm, extarrows, ulem}
\usepackage{amssymb}
\usepackage{nameref, hyperref}
 % \geometry{top=3cm, bottom=3cm, left=1.5cm, right=1.5cm}
\usepackage{algorithm}
\usepackage{algorithmic}
\usepackage{minted}
\usepackage{enumitem}
\setenumerate[1]{itemsep=0pt,partopsep=0pt,parsep=\parskip,topsep=5pt}
\setitemize[1]{itemsep=0pt,partopsep=0pt,parsep=\parskip,topsep=5pt}

% \usepackage{adjustbox}

\renewcommand\contentsname{Contents}

\tcbuselibrary{skins, breakable, theorems}

% \setlength{\leftskip}{10pt}
\setlength{\parindent}{10pt}
% \setlength{\parskip}{2em}
\renewcommand{\baselinestretch}{1.3}

\newcounter{RomanNumber}
\newcommand{\mrm}[1]{(\setcounter{RomanNumber}{#1}\Roman{RomanNumber})}

\newtcbtheorem[]{thm}{}
  {enhanced, theorem name and number, code={\edef\@currentlabelname{#2}}, 
  frame code={
        % \path[thick, draw] (frame.north west) -| (frame.north east) -| (frame.south east) -| (frame.south west) -| (frame.north west);
        \path[thick, draw] (frame.north west)  +(.5\baselineskip,0) -| +(0,-.5\baselineskip);
        % \path[thick, draw] (frame.north east) +(-.5\baselineskip,0) -| +(0,-.5\baselineskip);
        % \path[thick, draw] (frame.south west) +(.5\baselineskip,0) -| +(0,.5\baselineskip);
        \path[thick, draw] (frame.south east) +(-.5\baselineskip,0) -| +(0,.5\baselineskip);
    },
    left=1mm, right=1mm, top=1mm, bottom=1mm,
    colback=black!5,
    colframe=red!75!black,
    colbacktitle=black!0,
    coltitle=black!100,
    fonttitle=\bfseries}{thm}


\usepackage{xparse}
\NewDocumentEnvironment{qte}{m}{\begin{tcolorbox}[breakable, leftrule=2mm, rightrule=-0.1mm, toprule=-0.1mm, bottomrule=-0.1mm, arc=0mm, colframe=black!30!white, colback=white, coltext=white!50!black]}{\\\rightline{#1}\end{tcolorbox}}

\title{Algorithm Design and Analysis}
\date{\today}
\author{Not Strong Enough}

\begin{document}
\maketitle

\section{Homework}
\begin{thm}{Problem 4}{}
	[A Dynamic Programming Algorithm for the Binomial Coefficient] Using pseudocode, 
	write a dynamic programming algorithm computing  $\binom{n}{k}$.  Implement it in python!  What is it running 
	time in terms of $n$ and $k$? Would you say your algorithm is efficient?  Why or why not?
	
\end{thm}
\begin{algorithm}
	\caption{Caluculate Binomial Coefficient Using DP}
	\begin{algorithmic}
		\STATE Create a 2-dimension array $G$
		\FOR{$i=0$ to $n$}
		\STATE $G[i][0]=1$
		\ENDFOR
		\FOR{$i=0$ to $k$}
		\STATE $G[i][i]=1$
		\ENDFOR
		\FOR{$i=1$ to $n$}
		\FOR{$j=1$ to min($i-1$,$k$)}
		\STATE $G[i][j]=G[i-1][j]+G[i-1][j-1]$
		\ENDFOR
		\ENDFOR
		\RETURN $G[n][k]$
	\end{algorithmic}
\end{algorithm}
\begin{minted}{Python}
def calc_dp(n,k):
    arr = [[0 for i in range(k+1)] for j in range(n+1)]
    for i in range(n+1):
        arr[i][0]=1
    for i in range(k+1):
        arr[i][i]=1
    for i in range(1,n+1):
        for j in range(1,min(i-1,k)+1):
            arr[i][j]=arr[i-1][j]+arr[i-1][j-1]
return arr[n][k]
\end{minted}
complexity analysis: When we traverse the array and visit arr[i][j], we actually perform the add operation $\binom{i}{j}=\binom{i-1}{j}+\binom{i-1}{j-1}$. So the operation cost O($\log\binom{i}{j}$) time. Using the Stirling Formula, we can estimate
$\log\binom{i}{j}=(i+1/2)\log i-(i-j+1/2)\log (i-j)-(j+1/2)\log j =$ O($i$). The number of nodes we visit is O($k*(2n-k)/2$)=O($kn$). Since we will only visit arr[i][j] once, we can estimate the total complexity as below: the upper bound is $O(kn*n))$=(O$kn^2$), the lower bound is $\Omega(kn)$. The algorithm is efficient, because there is no redundant calculation.
\end{document}

