\documentclass[UTF8, a4paper, linespread=1.5]{article}

\usepackage{tcolorbox, listings, algorithm, minted, algpseudocode}
\usepackage{geometry, savesym, amsmath, enumerate, indentfirst, color, amsthm, bm, extarrows, ulem}
\usepackage{amssymb}
\usepackage{nameref, hyperref}
 \geometry{top=3cm, bottom=3cm, left=1.5cm, right=1.5cm}

\usepackage{enumitem}
\setenumerate[1]{itemsep=0pt,partopsep=0pt,parsep=\parskip,topsep=5pt}
\setitemize[1]{itemsep=0pt,partopsep=0pt,parsep=\parskip,topsep=5pt}

\renewcommand\contentsname{Contents}

\tcbuselibrary{skins, breakable, theorems}

% \setlength{\leftskip}{10pt}
\setlength{\parindent}{10pt}
% \setlength{\parskip}{2em}
\renewcommand{\baselinestretch}{1.3}

\newcounter{RomanNumber}
\newcommand{\mrm}[1]{(\setcounter{RomanNumber}{#1}\Roman{RomanNumber})}

\newtcbtheorem{thm}{}
  {enhanced, theorem name and number, code={\edef\@currentlabelname{#2}}, 
  frame code={
        % \path[thick, draw] (frame.north west) -| (frame.north east) -| (frame.south east) -| (frame.south west) -| (frame.north west);
        \path[thick, draw] (frame.north west)  +(.5\baselineskip,0) -| +(0,-.5\baselineskip);
        % \path[thick, draw] (frame.north east) +(-.5\baselineskip,0) -| +(0,-.5\baselineskip);
        % \path[thick, draw] (frame.south west) +(.5\baselineskip,0) -| +(0,.5\baselineskip);
        \path[thick, draw] (frame.south east) +(-.5\baselineskip,0) -| +(0,.5\baselineskip);
    },
    left=1mm, right=1mm, top=1mm, bottom=1mm,
    colback=black!5,
    colframe=red!75!black,
    colbacktitle=black!0,
    coltitle=black!100,
    fonttitle=\bfseries}{thm}


\usepackage{environ}
\RenewEnviron{math}{%
\begin{align*}
\BODY
\end{align*}
}

\title{CS217 -- Algorithm Design and Analysis \\ Homework 5}
\date{\today}
\author{Not Strong Enough}



\begin{document}
    \maketitle

    \begin{thm}{Exercise 1}{}
        Let $G = (V, c)$ be a flow network. Prove that flow is ``transitive'' in the following sense: if $r, s, t$ are vertices, and there is an $r$-$s$-flow of value $k$ and an $s$-$t$-flow of value $k$, then there is an $r$-$t$-flow of value $k$.
    \end{thm}

    \begin{proof}
        Note that there is an $r$-$s$-flow of value $k$ means that the value of the maximum $r$-$s$-flow is at least $k$, which also means that the value of the minimum $r$-$s$-cut is at least $k$. Similarly, the value of the minimum $s$-$t$-cut is also at least $k$.
        
        Now consider an $r$-$t$-cut. It is either an $r$-$s$-cut (if $s$ is not in the cut) or an $s$-$t$-cut (if $s$ is in the cut). So the capacity of the minimum $r$-$t$-cut is at least $k$. It follows that the value of the maximum $r$-$t$-flow is at least $k$, and thus there is an $r$-$t$-flow of value $k$.
    \end{proof}

    \newpage

    \begin{thm}{Exercise 3}{}
        Prove Menger’s Theorem. You have to prove two things: first, not both cases above can occur (this is rather easy); second, one of them must occur (this requires a tool from the lecture).
    \end{thm}
    \begin{proof}[Proof]
        Let $V(G) = \{v_1, \cdots \}$ and $s = v_p, t = v_q$. 
    
        We would like to construct a flow network $(V', s', t', c)$ where $V' = \{v_1, v_1', v_2, v_2', \cdots \}$ and
        $$c(u, v) = \begin{cases}
            1, \text{ if } \exists i, (u, v) = (v_i, v_i') \\
            \infty, \text{ if } \exists i, j, (u, v) = (v_i', v_j) \wedge (v_i, v_j) \in E \\
            0, \text{ otherwise }
        \end{cases}.$$
    
        And finally let $s' = v_p', t' = v_q$. Then:
    
        \begin{itemize}
            \item There are $k$ vertex disjoint paths $p_1, \cdots ,p_k$ in $G$ iff there is a flow $f$ in $(V', s', t', c)$ with $\text{val}(f) = k$, iff $\max \text{val}(f) \ge k$.
            \item There are $k-1$ vertices $v_{i_1}, \cdots ,v_{i{k-1}}$ in $V \setminus \{s, t\} $ such that $G - \{v_{i_1},\cdots , v_{i_{k-1}}\} $ contains no $s-t$ path iff there is a cut $S$ in $(V', s', t', c)$ with $\text{cap}(S) = k-1$ by making $v_{i_j} \in S$ and $v_{i_j}' \not\in S$ for all $j$, iff $\min \text{cap}(S) < k$.
        \end{itemize}
    
        By \textit{Max-Flow Min-Cut Theorem}, let $\max \text{val}(f) = \min \text{cap}(S) = l$. Then:
        \begin{itemize}
            \item Either $l \ge k$, resulting in $1$ holds while $2$ does not.
            \item Or $l < k$, resulting in $2$ holds while $1$ does not.
        \end{itemize}
    
        Therefore, \textit{exactly one} of the two statements is true.
    \end{proof}

    \newpage

    In the two exercises below, let $\Gamma(X)$ be the neighbors of $X$. 
    \begin{thm}{Exercise 4}{}
        Consider the induced bipartite subgraph $H_n[L_i\cup L_{i+1}]$, show that for $i<n/2$ the graph has a matching of size $|L_i|=\binom{n}{i}$
    \end{thm}
    \begin{proof}
        Use Hall's Theorem, the size of maximum matching equals $\min_{X\subseteq L_i}|L_i|-|X|+|\Gamma(X)|$. 

        Since in $H_n[L_i\cup L_{i+1}]$ the degree of each vertex in $L_i$ is $n-i$, 
        and that of each vertex in $L_{i+1}$ is $i+1$, there is $|X|(n-i)\leq|\Gamma(X)|(i+1)$. 
        As $i<n/2$, $|X|\leq|\Gamma(x)|\frac{i+1}{n-i}\leq|\Gamma(X)|$, and only if $|X|=0$ can the equality be achieved. 

        So there is $\min_{X\subseteq L_i}(|L_i|-|X|+|\Gamma(X)|)=|L_i|=\binom{n}{i}$. 
    \end{proof} 

    \newpage

    \begin{thm}{Exercise 5}{}
        Show that there are $\binom{n}{i}$ paths in $H_n$ starting at $L_i$ ending in $L_{n-i}$ and are disjoint. 
    \end{thm}
    \begin{proof}
        Construct the flow in following steps: 

        1. Split each vertex $v$ in $\bigcup_{i\leq k\leq n-i}L_k$ into two, $v_{in}$ and $v_{out}$, there is $(v_{in},v_{out})$ with capacity $1$ in the flow. 
        If there is an edge $(u,v)$ in $H_n$, then there is an edge $(u_out, v_in)$ with capacity $\infty$ in the flow. 

        2. Set a new point $s$ connected to $v_{i,in}$ for each $v_i\in L_i$ in the flow, the capacity is $1$. 
        Symmetrically, for each $v_{n-i}\in L_{n-i}$, there is an edge $(v_{n-i,out},t)$ with capacity $1$ in the flow. 

        Then there is a flow that, for edges from $s$ or end to $t$ the flow takes $1$; for edges in form of 
        $(v_{k,out},v_{k+1,in})$, the flow takes $\frac{\binom{n}{i}}{\binom{n}{k}(n-k)}$; 
        for edges in form of $(v_{k,in},v_{k,out})$, the flow takes $\frac{\binom{n}{i}}{\binom{n}{k}}$. 

        It is obvious that the flow is well-defined, that for each vertex (except for $s,t$) the flow in equals the flow out. 
        And the total flow is $\binom{n}{i}$. Besides, it is the maxflow since the flow out of $s$ is no more than $\binom{n}{i}$. 

        So there also exists a integer max flow $F$ whose size is $\binom{n}{i}$. 
        Since in the flow network, the capacity of $(v_{in},v_{out})$ for each $v$ is $1$, each path in the flow network is disjoint. 
        (Otherwise, the flow in for some vertex $v_{in}$ is greater than $1$). 
        The capacity of each path is $1$, so there are $\binom{n}{i}$ paths. 

        Combine $v_in$ and $v_out$ together and remove $s,t$, these $\binom{n}{i}$ paths turns to be disjoint paths from $L_i$ to $L_{n-i}$.
    \end{proof} 

    \newpage

    \begin{thm}{Exercise 6}{}
        Let $\nu(G)$ denote the size of a maximum matching of $G=(V,E)$.
        Show that a bipartite graph $G$ has at most $2^{\nu(G)}$ mimimum vertex covers.
    \end{thm}
    \begin{proof}
        From the K\"{o}nig's Theorem, we know that the size of mimimum vertex cover is $\nu(G)$ if $G$ is a bipartite graph.
        Let $C$ be a minimum vertex cover.
        Then we can construct new minimum vertex covers by choosing vertices from $C$ and $V\setminus C$.
        In other words, all minimum vertex covers can be represented by $X\cup Y$, where $X\subseteq C, Y \subseteq V\setminus C$.
        Denote $N(A)=\{b \mid \exists a\in A, \text{ there is an edge between } a \text{ and } b \}$.
        For all $X\subseteq C$, to construct a vertex cover, $Y$ must touch all edges touched by $C\setminus X$ but not by $X$.
        \begin{itemize}
            \item If $N(C\setminus X)\cap (C\setminus X)\ne \emptyset$, there does not exist such a $Y$.
            \item  If $N(C\setminus X)\cap (C\setminus X)= \emptyset$, then $Y$ must be at least $N(C\setminus X) \cap (V\setminus C)$ to be a vertex cover.
            To make $X\cap Y$ a minimum vertex cover, $Y$ has to be $N(C\setminus X) \cap (V\setminus C)$.
        \end{itemize}
        
        Since $X$ has $2^{\nu(G)}$ choices, $G$ has at most $2^{\nu(G)}$ mimimum vertex covers.
    \end{proof}

    \newpage

    Obviously, this is not true for general (non-bipartite) graphs: the triangle $K_3$ has $\nu(K_3) = 1$ but it has three minimum vertex covers. The five-cycle $C_5$ has $\nu(C_5) = 2$ but has five minimum vertex covers.

    \begin{thm}{Exercise 7}{}
        Is there a function $f \colon \mathbf{N_0} \rightarrow \mathbf{N_0}$ such that every graph with $\nu(G) = k$ has at most $f(k)$ minimum vertex covers? How small a function $f$ can you obtain?
    \end{thm}

    \begin{proof}[Solution]
        Suppose that we have a graph $G = (V, E)$ and one of its maximum matching $M \subseteq E$ with $|M| = \nu(G) = k$.
        
        We have the two following observations:
        
        \begin{itemize}
            \item For any vertex cover $C \subseteq V$ of $G$, for every edge in $M$ there must be at least one of its endpoint which is in $C$. Otherwise there exists an edge in $M$ such that neither of its endpoints is in $C$, which means that this edge is uncovered and therefore $C$ is not a vertex cover.
            \item For any vertex $v$ which is not matched, all of its neighbors must be matched, or the edge between $v$ and one of its unmatched neighbors can be added to the maximum matching and therefore $M$ is not maximum.
        \end{itemize}
        
        We now construct a vertex set $C_0 \subseteq V$ such that for every edge $(u, v)$ in $M$, either $u \in C_0$, or $v \in C_0$, or both $u, v \in C_0$. There are $3^k$ possible $C_0$ in total.
        
        For each possible $C_0$, note that it may not be a ``vertex cover'' by far. So we try to construct another vertex set $C_1$ from $C_0$. Let $C_1$ be an empty set at the beginning. From the second observation above, for every unmatched vertex $v$, there are two cases. If all of its neighbors are in $C_0$, then we do nothing, since every edge connected to $v$ is covered by vertices in $C_0$. Otherwise we add it into $C_1$ to cover the edges which $C_0$ didn't cover. Note that $C_1$ is {\it uniquely determined by $C_0$}.
        
        Let $\mathcal{C}$ be a family of vertex covers, which is initialized to empty. Now consider $C_0 \cup C_1$. We know that it is also uniquely determined by $C_0$. If it is a vertex cover, we add it to $\mathcal{C}$. There are at most $3^k$ vertex covers in $\mathcal{C}$, since there are $3^k$ possible $C_0$, and for some $C_0$ and its corresponding $C_1$, $C_0 \cup C_1$ may not be a vertex cover.
        
        Claim that any minimum vertex cover $C$ must belong to $\mathcal{C}$. Because from the first observation, we can let the unique $C_0$ be the matched vertices covered by $C$. And then unique $C_1$ can be constructed from $C_0$. $C_0 \cup C_1$ is the minimum vertex cover when $C_0$ is fixed. So $C = C_0 \cup C_1 \in \mathcal{C}$.
        
        So there are at most $3^k$ minimum vertex covers in total, and $f(k) = 3^k$. Also note that this upper bound is {\it tight}. Just consider the triangle $K_3$ --- it has $3 = 3^1 = 3^{\nu(K_3)}$ minimum vertex covers.
    \end{proof}

\end{document}