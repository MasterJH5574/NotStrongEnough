\documentclass[UTF8, a4paper, linespread=1.5]{article}

\usepackage{tcolorbox, listings, algorithm, minted, algpseudocode}
\usepackage{geometry, savesym, amsmath, enumerate, indentfirst, color, amsthm, bm, extarrows, ulem}
\usepackage{amssymb}
\usepackage{nameref, hyperref}
 \geometry{top=3cm, bottom=3cm, left=1.5cm, right=1.5cm}

\usepackage{enumitem}
\setenumerate[1]{itemsep=0pt,partopsep=0pt,parsep=\parskip,topsep=5pt}
\setitemize[1]{itemsep=0pt,partopsep=0pt,parsep=\parskip,topsep=5pt}

\renewcommand\contentsname{Contents}

\tcbuselibrary{skins, breakable, theorems}

% \setlength{\leftskip}{10pt}
\setlength{\parindent}{10pt}
% \setlength{\parskip}{2em}
\renewcommand{\baselinestretch}{1.3}

\newcounter{RomanNumber}
\newcommand{\mrm}[1]{(\setcounter{RomanNumber}{#1}\Roman{RomanNumber})}

\newtcbtheorem{thm}{}
  {enhanced, theorem name and number, code={\edef\@currentlabelname{#2}}, 
  frame code={
        % \path[thick, draw] (frame.north west) -| (frame.north east) -| (frame.south east) -| (frame.south west) -| (frame.north west);
        \path[thick, draw] (frame.north west)  +(.5\baselineskip,0) -| +(0,-.5\baselineskip);
        % \path[thick, draw] (frame.north east) +(-.5\baselineskip,0) -| +(0,-.5\baselineskip);
        % \path[thick, draw] (frame.south west) +(.5\baselineskip,0) -| +(0,.5\baselineskip);
        \path[thick, draw] (frame.south east) +(-.5\baselineskip,0) -| +(0,.5\baselineskip);
    },
    left=1mm, right=1mm, top=1mm, bottom=1mm,
    colback=black!5,
    colframe=red!75!black,
    colbacktitle=black!0,
    coltitle=black!100,
    fonttitle=\bfseries}{thm}


\usepackage{environ}
\RenewEnviron{math}{%
\begin{align*}
\BODY
\end{align*}
}

\title{CS217 -- Algorithm Design and Analysis \\ Homework 5}
\date{\today}
\author{Not Strong Enough}



\begin{document}

\begin{thm}{t8}{}
    On expection, how many comparisions will it make to use QUICKSELECT to find the minimum of the array. 
\end{thm}
    Using the analysis in exercise 9(see below)\ref{thm:t9}, we can get that 
    \begin{math}
        \sum_{1\leq i<j\leq n} B_{i,j,k}&= \sum_{i=1}^{k-1}\sum_{j=i+1}^{k} B_{i,j,k}+\sum_{i=k}^{n-1}\sum_{j=i+1}^{n} B_{i,j,k}\\
                                        &= k-H_k+n-(k+1)-H_{n-k+1}\\
                                        &= 1-H_1+n-2-H_n\\
                                        &= n - 2 - H_n\\
                                        &= n - \log(n) - 2 - o(1)\\
    .\end{math}

\begin{thm}{t9}{t9}
    Derive a formula for $\mathbb{E}_\pi[C(\pi, k)]$, up to additive terms of order $o(n)$. You might want to introduce $\kappa = k / n$.
\end{thm}

\begin{itemize}
    \item case $i<j\leq k$: 
        \begin{math}
            \sum_{i=1}^{k-1}\sum_{j=i+1}^{k} B_{i,j,k}&=\sum_{i=1}^{k-1}\frac{1}{k-i+1}*(k-i)\\
                                                      &=\sum_{i=1}^{k-1}(1-\frac{1}{k-i+1})=k-(1+\frac{1}{2}+\frac{1}{3}+\dots\frac{1}{k})\\
                                                      &= k - H_k
        .\end{math}
    \item case $k\leq i<j$:
        \begin{math}
            \sum_{i=k}^{n-1}\sum_{j=i+1}^{n} B_{i,j,k}&=\sum_{j=k+1}{n}\sum_{i=k}^{j-1} B_{i,j,k}\\
                                                      &=\sum_{j=k+1}^{n}\frac{j-k+1}*(j-k)\\
                                                      &=\sum_{j=k+1}^{n}(1-\frac{1}{j-k+1})\\
                                                      &=n-(k+1)-(1+\frac{1}{2}+\frac{1}{3}+\dots+\frac{1}{n-k+1})\\
                                                      &=n-(k+1)-H_{n-k+1}
        .\end{math}
\end{itemize}

\begin{itemize}
    \item case $i<k<j$:
        \begin{math}
            \sum _{i=1}^{k-1} \sum _{j=k+1}^n \frac{1}{j-i+1} &< \sum _{i=1}^{k-1} \int_{k+1}^{n+1} \frac{1}{j-i} \mathrm{d} j \\
                                              &= \sum _{i=1}^{k-1} \ln(n+1-i) - \ln(k+1-i) \\
                                              &< \int_{i=0}^{k-1} \ln(n+1-i) - \ln(k-i) \mathrm{d} i \\
                                              &= (n+1)\ln(n+1) - k\ln k - (n-k+2)\ln(n-k+2)
        .\end{math}

        On the other hand, similarly we can have
        \begin{math}
            \sum _{i=1}^{k-1} \sum _{j=k+1}^n \frac{1}{j-i+1} &> -(-k+n+2) \ln (-k+n+2)-(k+2) \ln (k+2)+(n+1) \ln (n+1)+3 \ln (3)
        .\end{math}

        The difference between the upper bound and the lower bound is $o(n)$, thus the inaccuracy is $o(n)$.

        To write this formula more simply, we have
        \begin{math}
            \sum _{i=1}^{k-1} \sum _{j=k+1}^n \frac{1}{j-i+1} = n \ln n - k \ln k - (n-k) \ln (n-k) + o(n)
        .\end{math}

        Summing three terms up,
        \begin{math}
            \mathbb{E}_\pi[C(\pi,k)] = n - 1 - H_k - H_{n-k+1} + n\ln n - k\ln k - (n-k)\ln (n-k) + o(n)
        .\end{math}

\end{itemize}

\end{document}
