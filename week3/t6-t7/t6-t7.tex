\documentclass[UTF8, a4paper, linespread=1.5]{article}

\usepackage{tcolorbox, listings, algorithm, minted, algpseudocode}
\usepackage{geometry, savesym, amsmath, enumerate, indentfirst, color, amsthm, bm, extarrows, ulem}
\usepackage{amssymb}
\usepackage{nameref, hyperref}
 \geometry{top=3cm, bottom=3cm, left=1.5cm, right=1.5cm}

\usepackage{enumitem}
\setenumerate[1]{itemsep=0pt,partopsep=0pt,parsep=\parskip,topsep=5pt}
\setitemize[1]{itemsep=0pt,partopsep=0pt,parsep=\parskip,topsep=5pt}

\renewcommand\contentsname{Contents}

\tcbuselibrary{skins, breakable, theorems}

% \setlength{\leftskip}{10pt}
\setlength{\parindent}{10pt}
% \setlength{\parskip}{2em}
\renewcommand{\baselinestretch}{1.3}

\newcounter{RomanNumber}
\newcommand{\mrm}[1]{(\setcounter{RomanNumber}{#1}\Roman{RomanNumber})}

\newtcbtheorem{thm}{}
  {enhanced, theorem name and number, code={\edef\@currentlabelname{#2}}, 
  frame code={
        % \path[thick, draw] (frame.north west) -| (frame.north east) -| (frame.south east) -| (frame.south west) -| (frame.north west);
        \path[thick, draw] (frame.north west)  +(.5\baselineskip,0) -| +(0,-.5\baselineskip);
        % \path[thick, draw] (frame.north east) +(-.5\baselineskip,0) -| +(0,-.5\baselineskip);
        % \path[thick, draw] (frame.south west) +(.5\baselineskip,0) -| +(0,.5\baselineskip);
        \path[thick, draw] (frame.south east) +(-.5\baselineskip,0) -| +(0,.5\baselineskip);
    },
    left=1mm, right=1mm, top=1mm, bottom=1mm,
    colback=black!5,
    colframe=red!75!black,
    colbacktitle=black!0,
    coltitle=black!100,
    fonttitle=\bfseries}{thm}


\usepackage{environ}
\RenewEnviron{math}{%
\begin{align*}
\BODY
\end{align*}
}

\title{CS217 -- Algorithm Design and Analysis \\ Homework 5}
\date{\today}
\author{Not Strong Enough}



\begin{document}
\maketitle

\begin{thm}{}{}
    Let $B_{i,j,k}$ be an indicator variable which is 1 if $i$ is a common ancestor of  $j$ and $k$ appear in the subtree $T(\pi)$ rooted at $i$.
    What is $\mathbb{E}[B_{i,j,k}]$? Give a succinct formula for this.
\end{thm}

\begin{proof}[Solution]
  	Suppose we are at node whose pivot is $p$. If $p< min(i,j,k)$ or $p>max(i,j,k)$, then $i,j,k$ are all in the left subtree or all in the right subtree, which means we have to check the subtree.
  	If $min(i,j,k)\le p \le max(i,j,k)$, we have to discuss some cases.
  	
  	case 1.if $p=i$, then $B_{i,j,k}=1$ because $i$ is the ancestor of $j$ and $k$, leading to $B_{i,j,k}=1$.
  	
  	case 2.if case 1 isn't satisfied and $p\ne j$ and $p\ne k$, then $i,j,k$ are not in the same subtree. So $B_{i,j,k}=0$.
  	
  	case 3.if case 1 and 2 isn't satisfied , then $j$ or $k$ is the ancestor of $i$, so $B_{i,j,k}=0$.
  	
  	In conclusion,$\mathbb{E}[B_{i,j,k}]=\frac{1}{max(i,j,k)-min(i,j,k)+1}$
\end{proof}



\begin{thm}{}{}
	Let $C(\pi,k)$ be the number of comparisons made by QUICKSELECT	when given $\pi$ as input. Design a formula for $C(\pi,k)$ with the help of the indicator variables $A_{i,j}$ and $B_{i,j,k}$ (analogous to the formula $\sum_{i\ne j}A_{i,j}$) for the number of comparisons made by quicksort).
	
\end{thm}

\begin{proof}[Solution]
	Observe that 2 number $i,j$ will be compared if and only if $j\ne i$ and $j$ and $k$ are in the subtree of $i$.
	
	So $C(\pi,k)=\sum_{i\ne j}B_{i,j,k}$.
\end{proof}



\end{document}

