%! Author = jinho
%! Date = 4/19/2020

\documentclass[UTF8, a4paper, linespread=1.5]{article}

\usepackage{tcolorbox, listings, algorithm, minted, algpseudocode}
\usepackage{geometry, savesym, amsmath, enumerate, indentfirst, color, amsthm, bm, extarrows, ulem}
\usepackage{amssymb}
\usepackage{nameref, hyperref}
 \geometry{top=3cm, bottom=3cm, left=1.5cm, right=1.5cm}

\usepackage{enumitem}
\setenumerate[1]{itemsep=0pt,partopsep=0pt,parsep=\parskip,topsep=5pt}
\setitemize[1]{itemsep=0pt,partopsep=0pt,parsep=\parskip,topsep=5pt}

\renewcommand\contentsname{Contents}

\tcbuselibrary{skins, breakable, theorems}

% \setlength{\leftskip}{10pt}
\setlength{\parindent}{10pt}
% \setlength{\parskip}{2em}
\renewcommand{\baselinestretch}{1.3}

\newcounter{RomanNumber}
\newcommand{\mrm}[1]{(\setcounter{RomanNumber}{#1}\Roman{RomanNumber})}

\newtcbtheorem{thm}{}
  {enhanced, theorem name and number, code={\edef\@currentlabelname{#2}}, 
  frame code={
        % \path[thick, draw] (frame.north west) -| (frame.north east) -| (frame.south east) -| (frame.south west) -| (frame.north west);
        \path[thick, draw] (frame.north west)  +(.5\baselineskip,0) -| +(0,-.5\baselineskip);
        % \path[thick, draw] (frame.north east) +(-.5\baselineskip,0) -| +(0,-.5\baselineskip);
        % \path[thick, draw] (frame.south west) +(.5\baselineskip,0) -| +(0,.5\baselineskip);
        \path[thick, draw] (frame.south east) +(-.5\baselineskip,0) -| +(0,.5\baselineskip);
    },
    left=1mm, right=1mm, top=1mm, bottom=1mm,
    colback=black!5,
    colframe=red!75!black,
    colbacktitle=black!0,
    coltitle=black!100,
    fonttitle=\bfseries}{thm}


\usepackage{environ}
\RenewEnviron{math}{%
\begin{align*}
\BODY
\end{align*}
}

\title{CS217 -- Algorithm Design and Analysis \\ Homework 5}
\date{\today}
\author{Not Strong Enough}



% Document
\begin{document}
    \begin{thm}{}{}
        Give an algorithm for MCP of running time $O(m\log\log m)$.
    \end{thm}
    \begin{proof}[Solution]
        \ 
        
        \begin{algorithm}
            \caption{An algorithm for MCP problem.}
            \begin{algorithmic}
                %\Require edges $e_1,\ldots,e_m$
                \Function{MCP}{$V, E = \{e_1, e_2, \ldots, e_m\}, c$}
                    \State $C\gets \{e_1,\ldots,e_m\}$
                    \State $oldEdges \gets \emptyset$
                    \State $d\gets 0$
                    \While{there are multiple capacities of the edges in $C$} \Comment{Loop 1}
                        \State $divisions\gets [C]$ \Comment Let $divisions$ be a list only containing the set $C$.
                        \For{$i=1$ to $2^{d}$} \Comment{Loop 2}
                        
                            \State Let $S_1, S_2, \ldots, S_k$ denote the edge sets in $divisions$ in order, i.e., representing $divisions$ as $[S_1, \ldots, S_k]$.
                            \State Use median-of-medians algorithm $k$ times to find the median capacity of the capacities of the edges in each set. Denote them as $m_1, m_2, \ldots, m_k$ respectively. (If there are two, choose the larger one.)
                            %\State $m_{1},\cdots,m_{k}\gets$ the median of $S_1,\cdots,S_k$ in means of weight
                            
                            \State $newDivisions \gets [\ ]$ \Comment Let $newDivisions$ be an empty list.
                            \For{$j = 1$ to $k$} \Comment{Loop 3}
                            \State Append $\{s\in S_{j}:c(s) \geqslant m_{j}\}$ to $newDivisions$.
                            \State Append $\{s\in S_{j}:c(s) < m_{j}\}$ to $newDivisions$.
                            %\State $divisions\gets \bigcup\limits_{j=1}^{k} \{s\in S_{j}:c(s)<=m_{j}\}\cup \{s\in S_{j}:c(s)>m_{j}\}$
                            \EndFor
                            \State $divisions \gets newDivisions$
                        \EndFor
                        \State $C, oldEdges\gets$ \Call{GetDivision}{$divisions$, $oldEdges$} %$GetDivision(divisions)$.
                    \State $d\gets 2^{d}+d$
                    \EndWhile
                    \State \Return the capacity of the edges in $C$
                \EndFunction
                
                \bigskip
                
                \Function{GetDivision}{$divisions$, $oldEdges$} \Comment Using the algorithm of exercise 1.
                    \State $G\gets (V, oldEdges)$
                    
                    \For {$div$ in $divisions$ in the order of the list} %(iterate from higher capacity to lower capacity)
                        \State Add all edges in $div$ to $G$.
                        \If{$t$ is now reachable from $s$}%\If{$G$ is s-t-connected}
                            \State \Return $div$, $oldEdges$
                        \Else
                            \State $oldEdges \gets oldEdges \cup div$
                        \EndIf
                    \EndFor
                \EndFunction
            \end{algorithmic}
        \end{algorithm}
    
        {\bf The pseudocode of the algorithm is in the next page.}
    
        The correctness of the algorithm comes from below.
        First, we initialize our $c^*$ candidates to be all capacities of the edges in $G$.
        Then, in each iteration, we divide it into many divisions of approximately the same size in order, and test which division $c^*$ is in.
        Finally, there will be only one candidate left, and the only candidate is $c^*$.

        Now we'll analyze the running time.
        Note that using median-of-medians algorithm to find the median of a set $S$ has running time $O(|S|)$.
        In ``Loop 2'' of function \textsc{MCP}, we find the median of $S_1,\ldots,S_k$ in $2^d$ iterations.
        Note that there is an invariant equation $\sum_{i=1}^{k}|S_k|=|C|$.
        So finding the median requires $O(|C|\cdot 2^{d})$ in total.
        The \textsc{GetDivision} function may need to add all the edges into the graph in the worst case.
        Testing the connectivity will use $O(m)$ time in total if we record a set of reachable vertices as decribed in Exercise 1.
        So the running time of \textsc{GetDivision} is $O(m)$.

        %Observe that $|C|\cdot 2^{d}=O(m)$.
        We now prove that $|C|\cdot 2^{d}=O(m)$ by induction.
        Initially, $|C|=m$ and $d=0$.
        It obviously holds.
        Suppose $|C_{old}|\cdot 2^{d_{old}}=O(m)$.
        We have $|C_{new}|=O(|C_{old}|\cdot (\frac{1}{2})^{2^{d_{old}}})$,
        $d_{new}=d_{old}+2^{d_{old}}$.
        So  $|C_{new}|\cdot 2^{d_{new}}=O(m)$

        Now we know that a single iteration of ``Loop 1'' takes time $O(m)$.
        Note that $2^{d}<m$.
        Since $d$ increases exponentially, ``Loop 1'' will iterate $O(\log^*m)$ times.
        Finally, we get the running time is $O(m\log^*m)$.
        Obviously this algorithm has a better running time than $O(m\log\log m),O(m\log\log\log m),O(m\log\log\log\log m)$ when $m$ is sufficiently large.
    \end{proof}

    \begin{thm}{}{}
        Give an algorithm for MCP that runs in $O(m\log\log\log m)$? How about $O(m\log\log\log\log m)$?
        How far can you get?
    \end{thm}
    \begin{proof}[Solution]
        See Problem 2.
    \end{proof}
\end{document}