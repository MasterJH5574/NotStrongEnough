	\documentclass[UTF8, a4paper, linespread=1.5]{article}

\usepackage{tcolorbox, listings}
\usepackage{geometry, savesym, amsmath, enumerate, indentfirst, color, amsthm, bm, extarrows, ulem}
\usepackage{amssymb}
\usepackage{nameref, hyperref}
 % \geometry{top=3cm, bottom=3cm, left=1.5cm, right=1.5cm}

\usepackage{enumitem}
\setenumerate[1]{itemsep=0pt,partopsep=0pt,parsep=\parskip,topsep=5pt}
\setitemize[1]{itemsep=0pt,partopsep=0pt,parsep=\parskip,topsep=5pt}

% \usepackage{adjustbox}

\renewcommand\contentsname{Contents}

\tcbuselibrary{skins, breakable, theorems}

% \setlength{\leftskip}{10pt}
\setlength{\parindent}{10pt}
% \setlength{\parskip}{2em}
\renewcommand{\baselinestretch}{1.3}

\newcounter{RomanNumber}
\newcommand{\mrm}[1]{(\setcounter{RomanNumber}{#1}\Roman{RomanNumber})}

\newtcbtheorem[]{thm}{}
  {enhanced, theorem name and number, code={\edef\@currentlabelname{#2}}, 
  frame code={
        % \path[thick, draw] (frame.north west) -| (frame.north east) -| (frame.south east) -| (frame.south west) -| (frame.north west);
        \path[thick, draw] (frame.north west)  +(.5\baselineskip,0) -| +(0,-.5\baselineskip);
        % \path[thick, draw] (frame.north east) +(-.5\baselineskip,0) -| +(0,-.5\baselineskip);
        % \path[thick, draw] (frame.south west) +(.5\baselineskip,0) -| +(0,.5\baselineskip);
        \path[thick, draw] (frame.south east) +(-.5\baselineskip,0) -| +(0,.5\baselineskip);
    },
    left=1mm, right=1mm, top=1mm, bottom=1mm,
    colback=black!5,
    colframe=red!75!black,
    colbacktitle=black!0,
    coltitle=black!100,
    fonttitle=\bfseries}{thm}


\usepackage{xparse}
\NewDocumentEnvironment{qte}{m}{\begin{tcolorbox}[breakable, leftrule=2mm, rightrule=-0.1mm, toprule=-0.1mm, bottomrule=-0.1mm, arc=0mm, colframe=black!30!white, colback=white, coltext=white!50!black]}{\\\rightline{#1}\end{tcolorbox}}

\title{Theory of Probability}
\date{\today}
\author{Not Strong Enough}

\begin{document}
\maketitle

\section{Homework}
\begin{thm}{Problem 22}{}
	 Let $f$ be a monotone function on $\mathbb{R}$ and let $A$ be the set of real numbers $x$ at which $f$ is discontinuous. Show that $|A| \le |\mathbb{N}|$
\end{thm}
\begin{proof}
	Without loss of generality, assume $f$ increases monotonically. Then for any $x \in A$, there should be $ f_+(x) > f_-(x)$. Because $f$ is increasing, for any $x_1 < x_2$, $ f_+(x_1) < f_-(x_2)$. Then we can construct a set $E=\{(f_-(x),f_+(x)) | x\in A\}$, whose elements are intervals that do not intersect with each other. Obviously, $|E|=|A|$. Due to the density of rational numbers, there are infinite rational numbers in each interval. Using the axiom of choice, we can bind each interval with a rational number in the interval, which forms a injection from $E$ to $\mathbb{Q}$. So $|A|=|E|\le|\mathbb{Q}|=|\mathbb{N}|$.
\end{proof}

	

\end{document}

