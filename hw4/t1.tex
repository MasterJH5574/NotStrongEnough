\documentclass[UTF8, a4paper, linespread=1.5]{article}

\usepackage{tcolorbox, listings, algorithm, minted, algpseudocode}
\usepackage{geometry, savesym, amsmath, enumerate, indentfirst, color, amsthm, bm, extarrows, ulem}
\usepackage{amssymb}
\usepackage{nameref, hyperref}
 \geometry{top=3cm, bottom=3cm, left=1.5cm, right=1.5cm}

\usepackage{enumitem}
\setenumerate[1]{itemsep=0pt,partopsep=0pt,parsep=\parskip,topsep=5pt}
\setitemize[1]{itemsep=0pt,partopsep=0pt,parsep=\parskip,topsep=5pt}

\renewcommand\contentsname{Contents}

\tcbuselibrary{skins, breakable, theorems}

% \setlength{\leftskip}{10pt}
\setlength{\parindent}{10pt}
% \setlength{\parskip}{2em}
\renewcommand{\baselinestretch}{1.3}

\newcounter{RomanNumber}
\newcommand{\mrm}[1]{(\setcounter{RomanNumber}{#1}\Roman{RomanNumber})}

\newtcbtheorem{thm}{}
  {enhanced, theorem name and number, code={\edef\@currentlabelname{#2}}, 
  frame code={
        % \path[thick, draw] (frame.north west) -| (frame.north east) -| (frame.south east) -| (frame.south west) -| (frame.north west);
        \path[thick, draw] (frame.north west)  +(.5\baselineskip,0) -| +(0,-.5\baselineskip);
        % \path[thick, draw] (frame.north east) +(-.5\baselineskip,0) -| +(0,-.5\baselineskip);
        % \path[thick, draw] (frame.south west) +(.5\baselineskip,0) -| +(0,.5\baselineskip);
        \path[thick, draw] (frame.south east) +(-.5\baselineskip,0) -| +(0,.5\baselineskip);
    },
    left=1mm, right=1mm, top=1mm, bottom=1mm,
    colback=black!5,
    colframe=red!75!black,
    colbacktitle=black!0,
    coltitle=black!100,
    fonttitle=\bfseries}{thm}


\usepackage{environ}
\RenewEnviron{math}{%
\begin{align*}
\BODY
\end{align*}
}

\title{CS217 -- Algorithm Design and Analysis \\ Homework 5}
\date{\today}
\author{Not Strong Enough}



\begin{document}
    \maketitle

    \begin{thm}{}{}
        Suppose the edges $e_1,\dots, e_m$ are sorted by their cost. Show how to solve MCP in time $O(n+m)$
    \end{thm}
    \begin{proof}[Solution]
        The algorithm is executed in iterations. Suppose that $c(e_1)\geq c(e_2)\geq\dots c(e_m)$. In iteration $i$, consider $e_i$. 

        There is a set of reachable vertices, and the initial status of the set is $s$. 
        Each vertex has an predecessor recording the predecessor in the path by which $s$ reaches the vertex. Its initial status is $null$. 
        Each vertex also has a set of unresolved edges whose start point is the vertex. Its initial status is $\emptyset$

        In each iteration, denote the handled edge by $e=(u,v)$. 

        If $u$ is not reachable, simply add $e$ into the unresolved edge set of $u$. 

        If $u,v$ are already reachable, nothing needs to be done. 
       
        If $u$ is already reachable but $v$ is not, set $v$ reachable and set its precursor be $u$. 
        Then handle all resolved edges of $v$. (let the target of each edge be reachable if it is not)
        Since this may introduce new reachable vertices, handle them too. This procedure is like a BFS. 

        Once $t$ is reached, the algorithm terminates, and the cost of the newly introduced edge is the cost of MCP. 
        Using the predecessor of each vertex, the path can be generated. 
    \end{proof}

        First prove that each vertex in the reachable set is really reachable(so it is well-defined): 
    \begin{proof}[Proof]
        The reachable set has the property that, except for $s$, all vertices in the set have a predecessor also in the set. 
        
        This is trivial since when we add an vertex to the reachable set, we set its precursor by a reachable vertex. 

        Using this property, since its precursor is reachable, the vertex itself is reachable as well. 
    \end{proof}

        Then prove that all reachable vertices using $\{e_1,\dots e_n\}$ are in reachable set after iteration $n$: 
    \begin{proof}[Proof]
        Assume that there is a path from $s$ to $v$ using edges $e_{i_1},e_{i_2}\dots e_{i_k},(i_1,i_2\dots i_k\in \{1,2,\dots n\})$. 
        And $\forall j\in \{1,2\dots k\},e_{i_j}=(u_j,u_{j+1});u_1=s,u_{k+1}=v$. 
        Let $M_j=\max\{i_1,i_2\dots i_j\}$, $m_j=\min\{i_1,i_2\dots i_j\}$

        Assume that after iteration $M_{q-1}$ $u_1,u_2\dots u_q$ are in reachable set and $u_q$ is added in iteration $M_{q-1}$. 
        If $i_q<M_{q-1}$, $e_q$ is in unresolved set of $u_q$. 
        So in iteration $M_{q-1}=M_q$, when adding $u_q$ in reachable set, 
        all unresolved edges of $u_q$ are handled so $u_{q+1}$ is added as well. 
        If $i_q>M_{q-1}$, in iteration $M_q=i_q$, $u_q$ is in reachable set so $u_{q+1}$ is added in reachable set. 

        Consider the initial status that after iteration $M_1=i_1$, $u_1=s,u_2$ are in reachable set and $u_2$ is added in iteration $M_1$. 

        By induction, it is proved that after iteration $M_k$, $u_1,u_2\dots u_{k+1}$ are all in reachable set. 
    \end{proof}
        Now it is trivial to prove the correctness of the algorithm: 
        
        If the cost of MCP is $c=e_n$, then the path only use edges in $\{e_1,\dots e_n\}$, so exactly in iteration $n$, $t$ is reachable. 

        Then prove the complexity of the algorithm is $O(n+m)$:
    \begin{proof}[Proof]
        Consider an arbitrary edge $e_n=(u,v)$. If in iteration $n$, $u$ is already reachable, 
        it is not added into the unresolved set so it will not visited. 
        Otherwise, it is added into the unresolved set of $u$, 
        and when $u$ is added into the reachable set, it will be visited. 

        Since $u$ can only added to reachable set once, $e_n$ is visited at most twice. So the complexity is $O(m)$

        In the initial step, each vertex is initialized, so the complexity is $O(n)$. 

        Combine the two steps together, the complexity is $O(n+m)$. 
    \end{proof} 

\end{document}

