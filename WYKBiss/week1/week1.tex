\documentclass[UTF8, a4paper, linespread=1.5]{article}

\usepackage{tcolorbox, listings, algorithm, minted, algorithmic}
\usepackage{geometry, savesym, amsmath, enumerate, indentfirst, color, amsthm, bm, extarrows, ulem}
\usepackage{amssymb, xeCJK}
\usepackage{nameref, hyperref}
 % \geometry{top=3cm, bottom=3cm, left=1.5cm, right=1.5cm}

\usepackage{enumitem}
\setenumerate[1]{itemsep=0pt,partopsep=0pt,parsep=\parskip,topsep=5pt}
\setitemize[1]{itemsep=0pt,partopsep=0pt,parsep=\parskip,topsep=5pt}

% \usepackage{adjustbox}

\renewcommand\contentsname{Contents}

\tcbuselibrary{skins, breakable, theorems}

% \setlength{\leftskip}{10pt}
\setlength{\parindent}{10pt}
% \setlength{\parskip}{2em}
\renewcommand{\baselinestretch}{1.3}

\newcounter{RomanNumber}
\newcommand{\mrm}[1]{(\setcounter{RomanNumber}{#1}\Roman{RomanNumber})}

\newtcbtheorem{thm}{}
  {enhanced, theorem name and number, code={\edef\@currentlabelname{#2}}, 
  frame code={
        % \path[thick, draw] (frame.north west) -| (frame.north east) -| (frame.south east) -| (frame.south west) -| (frame.north west);
        \path[thick, draw] (frame.north west)  +(.5\baselineskip,0) -| +(0,-.5\baselineskip);
        % \path[thick, draw] (frame.north east) +(-.5\baselineskip,0) -| +(0,-.5\baselineskip);
        % \path[thick, draw] (frame.south west) +(.5\baselineskip,0) -| +(0,.5\baselineskip);
        \path[thick, draw] (frame.south east) +(-.5\baselineskip,0) -| +(0,.5\baselineskip);
    },
    left=1mm, right=1mm, top=1mm, bottom=1mm,
    colback=black!5,
    colframe=red!75!black,
    colbacktitle=black!0,
    coltitle=black!100,
    fonttitle=\bfseries}{thm}


\usepackage{xparse}
\NewDocumentEnvironment{qte}{m}{\begin{tcolorbox}[breakable, leftrule=2mm, rightrule=-0.1mm, toprule=-0.1mm, bottomrule=-0.1mm, arc=0mm, colframe=black!30!white, colback=white, coltext=white!50!black]}{\\\rightline{#1}\end{tcolorbox}}

\title{Discussion on some exercises of Basic Probability Theory}
\date{\today}
\author{庄永昊 518030910440 \\ 金弘义 518030910333 \\ 赖睿航 518030910422 \\ 董海辰 518030910417}

\begin{document}
\maketitle

\begin{thm}{Exercise 21. Number Guessing}{}
A person writes two distinct integers on two cards, one per card,
and puts them on the table face down. Pick either of the two, look at it, and then guess
whether the other number is larger or smaller. Suppose that you have a good random number
generator. Prove that you have a strategy to make a correct guess with probability strictly
greater than $\frac{1}{2}$.
\end{thm}

\begin{proof}[Proof]
    Let the two distinct integers on cards be $p, q \in \mathbb{Z} $.

    The strategy exists if we can generate a random number on $\mathbb{Z} $. Let this random number be $n$.
    Our strategy is as follow: If $n > p$, we make the guess that the other number is greater.

    Suppose $p < q$, there will be six situations:

    \begin{table}[h!]
        \centering
        \begin{tabular}{lll}
            Seen number   & $p$     & $q$     \\
            $n < p$         & Wrong   & Correct \\
            $p < n < q$ & Correct & Correct \\
            $q < n$     & Correct & Wrong
        \end{tabular}
    \end{table}

    And we have the same probability to see $p$ and $q$. Thus the probability to guess correctly is
    $$p = \frac{1}{2} + \frac{P(p < n < q)}{2}$$

    , which is strictly greater that $\frac{1}{2}$.

    For example, if we can get a random angle $\theta \in [0,2\pi)$, and generate by $n = \cot \frac{\theta}{2}$. Then $$p = \frac{1}{2} + \frac{\cot^{-1} q - \cot^{-1} p}{2\pi}.$$

    And if we only have a coin, we can toss it for $k$ times to generate a random angle $k \in [0, 2\pi]$ by
    $k = \frac{\text{ integer represented by the sequence }}{2^k} \cdot 2\pi$. Then apply the procedure above.
\end{proof}

\begin{thm}{Exercise 22. Countable Discontinuous Point}{}
	 Let $f$ be a monotone function on $\mathbb{R}$ and let $A$ be the set of real numbers $x$ at which $f$ is discontinuous. Show that $|A| \le |\mathbb{N}|$
\end{thm}

\begin{proof}
	Without loss of generality, assume $f$ increases monotonically. Then for any $x \in A$, there should be $ f_+(x) > f_-(x)$. Because $f$ is increasing, for any $x_1 < x_2$, $ f_+(x_1) < f_-(x_2)$. Then we can construct a set $E=\{(f_-(x),f_+(x)) | x\in A\}$, whose elements are intervals that do not intersect with each other. Obviously, $|E|=|A|$. Due to the density of rational numbers, there are infinite rational numbers in each interval. Using the axiom of choice, we can bind each interval with a rational number in the interval, which forms a injection from $E$ to $\mathbb{Q}$. So $|A|=|E|\le|\mathbb{Q}|=|\mathbb{N}|$.
\end{proof}

\begin{thm}{Exercise 24. Chain \& Antichain}{}
   A chain in $2^\mathbb{N} $ is a subset of $2^\mathbb{N} $ of which any two elements are comparable. An antichain in $2^\mathbb{N} $ is a subset of $2^\mathbb{N} $ of which any two elements are incomparable. Can there be a chain whose length is $|2^\mathbb{N} |$? Can there be an antichain whose size is $|2^\mathbb{N} |$? 
\end{thm}
1. chain:

First we should claim that in all chains we consider, all elements have an immediate precursor. 
This can

We define max-chain $\mathfrak C$ on $\mathbb N$ be the following kind of chain: 

It starts from $\{x_0\}(x_0\in\mathbb N)$, then each element$C_\alpha\in\mathfrak C$ of the chain has 
exactly one element(which is a number) than the immediate precursor$C_\beta\in\mathfrak C$. 
Moreover, every max-chain has maximum element $\mathbb N$. 

For each max-chain$\mathfrak C$, we can construct a bijection $f$ that: $f(C_\alpha)=$the element newly 
introduced into $C_\alpha$. Since in any two elements in a set are different, 
an element can and can only be introduced once. So $f$ is a bijection, 
which means the length of a max-chain equals $\mid\mathbb N\mid$. 

The second step is to prove that any chain on $2^{\mathbb N}$ is a subchain of a max-chain. This is trivial: 

If one element$C_\alpha$ in the chain has more than one element which its immediate precursor $C_\beta$ 
does not have, we can put some more elements $C_\gamma, C_\theta\dots$between these two element, and 
$C_\gamma$ has exactly one more element than $C_\beta$, $C_\theta$ has exactly one more element than $C_\gamma$...

In the end, by adding new elements, a chain will be extended to a max-chain, whose length equals $\mid\mathbb N\mid$. 
So the cardnality of the orginal chain is at most $\mid\mathbb N\mid$. 

2: antichain:

Construct an antichain $\mathfrak A$ by the following step:

Devide numbers by pairs: $(0, 1);(2, 3);\dots (2n, 2n - 1);\dots$. Denote the pair $(2i, 2i + 1)$ by $p_i$.

Each set $A\in\mathfrak A$ contains one and only one element in each pair, making them different with each other:

For each two set, they contians different number at least in one pair, so they are uncomparable.
So $\mathfrak A$ is an antichain.

Now prove that $\mid\mathfrak A\mid=\mid2^{\mathbb N}\mid$:

We construct such a function $f:\mathfrak A\rightarrow 2^{\mathbb N}$
$$f(A)=\{\text{if $A$ contains $2i$, $i$ is in the set, otherwise(contains $2i+1$) $i$ is not in the set}\}$$

It is apparently that $f$ is a bijection between $\mathfrak A$ and $2^{\mathbb N}$.


\begin{thm}{Exercise 26. Liar Paradox}{}
    We came across two islanders A and B on Crete Island. "All Cretans are liars.", A said. "I'm lying", B said. Which statement is a paradox?
\end{thm}

\begin{proof}
    For the first statement, assume what A said is correct. Then all Cretans are liars, and what A said should be wrong. This cannot happen since we assume A didn't lie. However, if we assume what A said was wrong(i.e A is a liar), it follows that at least one Cretan who is not a liar. This doesn't contradict with the hypothesis. \textbf{Thus, from what A said, we can conclude that A is a liar. And A's statement is not a paradox.}

    For the second statement, if we assume B told the truth, then from the former analysis we know that B is a liar, which leads to a contradiction. Likewise, if we assume what B said is wrong, it follows that B didn't lies, which also leads to a contradiction. \textbf{So B's statement is a paradox.}

    The famous "liar paradox" is exactly the statement which B said. It is one of the self-referential paradoxes.
\end{proof}

\end{document}

