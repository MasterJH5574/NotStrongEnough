% This is a template for lecture notes.
\documentclass{article}
\usepackage[UTF8]{ctex}
\usepackage{amssymb}
\usepackage{amsmath}
\usepackage{amsthm}
\usepackage{geometry}
\usepackage{booktabs}
\usepackage{bm}
\usepackage{tcolorbox}
\usepackage{xunicode, mathrsfs, xltxtra, amsfonts, caption, latexsym}
\CTEXoptions[today=old]
%Some commonly used notations
%\geometry{a4paper,bottom = 3cm,left = 3cm, right = 3cm}

%for reference
\usepackage{hyperref}
\usepackage[capitalise]{cleveref}
\crefname{enumi}{}{}

\newtheorem{theorem}{Theorem}
\newtheorem{lemma}[theorem]{Lemma}
\newtheorem{proposition}[theorem]{Proposition}
\newtheorem{corollary}[theorem]{Corollary}
\newtheorem{fact}[theorem]{Fact}
\newtheorem{definition}[theorem]{Definition}
\newtheorem{remark}[theorem]{Remark}
\newtheorem{question}[theorem]{Question}
\newtheorem{answer}[theorem]{Answer}
\newtheorem{exercise}[theorem]{Exercise}
\newtheorem{example}[theorem]{Example}
%\newenvironment{proof}{\noindent \textbf{Proof:}}{$\Box$}
\newtheorem{observation}[theorem]{Observation}

%this is how we define operators.
\DeclareMathOperator{\rank}{rank} % rank

\newenvironment{myproof}{\ignorespaces\paragraph{Proof:}}{\hfill $\square$\par\noindent}

\title{Probability, Week 1, exerciese 24}
\author{Not Strong Enough}
\date{\today}
\def\mfa{\mathfrak A}

\begin{document}
\maketitle

1. chain: 

For each chain $\mathfrak C$, since each two elements in $\mathfrak C$ are comparable, they cannot have same size. 
So we can have a function $f:\mathfrak C\rightarrow\mathbb N:f(C)=$ the size of $C$, which is an injection. 

So there is $\mid\mathfrak C\mid\leq\mid\mathbb N\mid<\mid 2^{\mathbb N}\mid$. 

2: antichain: 

Construct an antichain $\mathfrak A$ by the following step: 

Devide numbers by pairs: $(0, 1);(2, 3);\dots (2n, 2n - 1);\dots$. Denote the pair $(2i, 2i + 1)$ by $p_i$. 

Each set $A\in\mathfrak A$ contains one and only one element in each pair, making them different with each other: 

For each two set, they contians different number at least in one pair, so they are uncomparable. 
So $\mathfrak A$ is an antichain. 

Now prove that $\mid\mathfrak A\mid=\mid2^{\mathbb N}\mid$: 

We construct such a function $f:\mathfrak A\rightarrow 2^{\mathbb N}$
$$f(A)=\{\text{if $A$ contains $2i$, $i$ is in the set, otherwise(contains $2i+1$) $i$ is not in the set}\}$$

It is apparently that $f$ is a bijection between $\mathfrak A$ and $2^{\mathbb N}$. 

\end{document}