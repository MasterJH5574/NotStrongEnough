\documentclass[UTF8, a4paper, linespread=1.5]{article}

\usepackage{tcolorbox, listings}
\usepackage{geometry, savesym, amsmath, enumerate, indentfirst, color, amsthm, bm, extarrows, ulem}
\usepackage{amssymb}
\usepackage{nameref, hyperref}
 % \geometry{top=3cm, bottom=3cm, left=1.5cm, right=1.5cm}

\usepackage{enumitem}
\setenumerate[1]{itemsep=0pt,partopsep=0pt,parsep=\parskip,topsep=5pt}
\setitemize[1]{itemsep=0pt,partopsep=0pt,parsep=\parskip,topsep=5pt}

% \usepackage{adjustbox}

\renewcommand\contentsname{Contents}

\tcbuselibrary{skins, breakable, theorems}

% \setlength{\leftskip}{10pt}
\setlength{\parindent}{10pt}
% \setlength{\parskip}{2em}
\renewcommand{\baselinestretch}{1.3}

\newcounter{RomanNumber}
\newcommand{\mrm}[1]{(\setcounter{RomanNumber}{#1}\Roman{RomanNumber})}

\newtcbtheorem[number within=subsection]{thm}{}
  {enhanced, theorem name and number, code={\edef\@currentlabelname{#2}}, 
  frame code={
        % \path[thick, draw] (frame.north west) -| (frame.north east) -| (frame.south east) -| (frame.south west) -| (frame.north west);
        \path[thick, draw] (frame.north west)  +(.5\baselineskip,0) -| +(0,-.5\baselineskip);
        % \path[thick, draw] (frame.north east) +(-.5\baselineskip,0) -| +(0,-.5\baselineskip);
        % \path[thick, draw] (frame.south west) +(.5\baselineskip,0) -| +(0,.5\baselineskip);
        \path[thick, draw] (frame.south east) +(-.5\baselineskip,0) -| +(0,.5\baselineskip);
    },
    left=1mm, right=1mm, top=1mm, bottom=1mm,
    colback=black!5,
    colframe=red!75!black,
    colbacktitle=black!0,
    coltitle=black!100,
    fonttitle=\bfseries}{thm}


\usepackage{xparse}
\NewDocumentEnvironment{qte}{m}{\begin{tcolorbox}[breakable, leftrule=2mm, rightrule=-0.1mm, toprule=-0.1mm, bottomrule=-0.1mm, arc=0mm, colframe=black!30!white, colback=white, coltext=white!50!black]}{\\\rightline{#1}\end{tcolorbox}}

\title{Probability Theory}
\date{\today}
\author{Not Strong Enough}

\begin{document}
\maketitle

\begin{thm}{Exercise 26.}{}
    We came across two islanders A and B on Crete Island. "All Cretans are liars.", A said. "I'm lying", B said. Which statement is a paradox?
\end{thm}

\begin{proof}
    For the first statement, assume what A said is correct. Then all Cretans are liars, and what A said should be wrong. This cannot happen since we assume A didn't lie. However, if we assume what A said was wrong(i.e A is a liar), it follows that at least one Cretan who is not a liar. This doesn't contradict with the hypothesis. \textbf{Thus, from what A said, we can conclude that A is a liar. And A's statement is not a paradox.}
    
    For the second statement, if we assume B told the truth, then from the former analysis we know that B is a liar, which leads to a contradiction. Likewise, if we assume what B said is wrong, it follows that B didn't lies, which also leads to a contradiction. \textbf{So B's statement is a paradox.}
    
    The famous "liar paradox" is exactly the statement which B said. It is one of the self-referential paradoxes.
\end{proof}

\end{document}

