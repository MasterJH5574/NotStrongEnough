\documentclass[UTF8, a4paper, linespread=1.5]{article}

\usepackage{tcolorbox, listings, algorithm, minted, algpseudocode}
\usepackage{geometry, savesym, amsmath, enumerate, indentfirst, color, amsthm, bm, extarrows, ulem}
\usepackage{amssymb}
\usepackage{nameref, hyperref}
 \geometry{top=3cm, bottom=3cm, left=1.5cm, right=1.5cm}

\usepackage{enumitem}
\setenumerate[1]{itemsep=0pt,partopsep=0pt,parsep=\parskip,topsep=5pt}
\setitemize[1]{itemsep=0pt,partopsep=0pt,parsep=\parskip,topsep=5pt}

\renewcommand\contentsname{Contents}

\tcbuselibrary{skins, breakable, theorems}

% \setlength{\leftskip}{10pt}
\setlength{\parindent}{10pt}
% \setlength{\parskip}{2em}
\renewcommand{\baselinestretch}{1.3}

\newcounter{RomanNumber}
\newcommand{\mrm}[1]{(\setcounter{RomanNumber}{#1}\Roman{RomanNumber})}

\newtcbtheorem{thm}{}
  {enhanced, theorem name and number, code={\edef\@currentlabelname{#2}}, 
  frame code={
        % \path[thick, draw] (frame.north west) -| (frame.north east) -| (frame.south east) -| (frame.south west) -| (frame.north west);
        \path[thick, draw] (frame.north west)  +(.5\baselineskip,0) -| +(0,-.5\baselineskip);
        % \path[thick, draw] (frame.north east) +(-.5\baselineskip,0) -| +(0,-.5\baselineskip);
        % \path[thick, draw] (frame.south west) +(.5\baselineskip,0) -| +(0,.5\baselineskip);
        \path[thick, draw] (frame.south east) +(-.5\baselineskip,0) -| +(0,.5\baselineskip);
    },
    left=1mm, right=1mm, top=1mm, bottom=1mm,
    colback=black!5,
    colframe=red!75!black,
    colbacktitle=black!0,
    coltitle=black!100,
    fonttitle=\bfseries}{thm}


\usepackage{environ}
\RenewEnviron{math}{%
\begin{align*}
\BODY
\end{align*}
}

\title{CS217 -- Algorithm Design and Analysis \\ Homework 5}
\date{\today}
\author{Not Strong Enough}



\begin{document}

    \begin{thm}{}{}
        Let $d$ be the shortest path distance from $s$ to $t$ in the directed graph $G$, where distance means sum of the $c(e)$ along the path. Show that opt($MCF$) $= d$.
    \end{thm}

    \begin{proof}
        First we show that opt($MCF$) $\leqslant d$. This is quite easy. Suppose $(v_1, \ldots v_k)$ where $k \geqslant 1$ and $(v_i, v_{i + 1}) \in E, v_1 = s, v_k = t$ is a shortest path in graph $G$. Then we can write a feasible solution to MCF by setting all $f(v_i, v_{i + 1}) = 1$ and setting the flow of edges which are not in the shortest path to $0$. (Obviously the two constraints of MCF are both satisfied.) Note that the value of this solution is exactly $d$, since $\sum_{e \in E} c(e) f(e) = \sum_{i = 1}^{k - 1} (c(v_i, v_{i + 1}) \cdot 1) = d$. Hence opt($MCF$) $\leqslant d$.
        
        Next we show that opt($MCF$) $\geqslant d$. We prove by contradiction. Assume that opt($MCF$) $< d$. Suppose we have a feasible solution $f^*$ with ${\rm val}(f^*) < d$. The idea is first to split the flow $f^*$ into some {\it single-path-flows}. And then we will show that among such $s-t$-paths, at least one path $P^*$ satisfies that $\sum_{e \in P^*} c(e) < d$. So $d$ is not the shortest path distance from $s$ to $t$, and it leads to a contradiction.
        
        \bigskip
        
        Now we focus on how to split $f^*$ into multiple single-path-flows. We first choose an edge $(u, v)$ such that $f(u, v)$ is non-negative and minimum among all edges in $G$. Since $f(u, v) \neq 0$ is minimum, we can always find a path from $s$ to $u$ and a path from $v$ to $t$ such that the flows of all the edges in both paths are at least $f(u, v)$ (e.g., using DFS or BFS). Let the two paths be $P_u$ and $P_v$. Connecting $P_u$, $P_v$ with $(u, v)$ we get a new path $P = \left(P_u, (u, v), P_v\right)$, with the flows of all edges on this path being at least $f(u, v)$ and the flow contribution of this path being exactly $f(u, v)$.
        
        So we can ``extract'' this path from $G$ by decreasing the flow of edges in this path by $f(u, v)$. Note that edge $(u, v)$ now has flow $0$. After extraction, we get a new graph $G'$ and a new ``flow function'' ${f^*}'$. We can validate that ${f^*}'$ is still a real flow by checking the flow-conservative contraint: the inflow decrease and outflow decrease of every vertices on path $P$ are always the same, and the extraction does not influence the inflow and outflow of vertices not on $P$. So ${f^*}'$ is a real flow.
        
        Repeat the ``find minimum and then extract'' process on the new graph. Note that each time the number of edges with non-negative flow is decreased by $1$, and the total number of edges in a graph is finite, so the process will surely terminate. Finally we get many single-path-flows. We claim that among these single-path-flows, at least one path $P^*$ satisfies that $\sum_{e \in P^*} c(e) < d$. Otherwise every single-path-flow has sum of $c(e)$ along the path being at least $d$, and it follows that ${\rm val}(f) \geqslant d$, which contradicts with our assumption. However, $\sum_{e \in P^*} c(e) < d$ is impossible since $d$ is the shortest minimum path distance, which also contradicts with this premise. So we have opt($MCF$) $\geqslant d$.
        
        \bigskip
        
        Combining opt($MCF$) $\leqslant d$ and opt($MCF$) $\geqslant d$ we can finally conclude that opt($MCF$) $= d$.
    \end{proof}


\end{document}

