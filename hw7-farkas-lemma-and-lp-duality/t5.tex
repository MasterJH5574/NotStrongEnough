\documentclass[UTF8, a4paper, linespread=1.5]{article}

\usepackage{tcolorbox, listings, algorithm, minted, algpseudocode}
\usepackage{geometry, savesym, amsmath, enumerate, indentfirst, color, amsthm, bm, extarrows, ulem}
\usepackage{amssymb}
\usepackage{nameref, hyperref}
 \geometry{top=3cm, bottom=3cm, left=1.5cm, right=1.5cm}

\usepackage{enumitem}
\setenumerate[1]{itemsep=0pt,partopsep=0pt,parsep=\parskip,topsep=5pt}
\setitemize[1]{itemsep=0pt,partopsep=0pt,parsep=\parskip,topsep=5pt}

\renewcommand\contentsname{Contents}

\tcbuselibrary{skins, breakable, theorems}

% \setlength{\leftskip}{10pt}
\setlength{\parindent}{10pt}
% \setlength{\parskip}{2em}
\renewcommand{\baselinestretch}{1.3}

\newcounter{RomanNumber}
\newcommand{\mrm}[1]{(\setcounter{RomanNumber}{#1}\Roman{RomanNumber})}

\newtcbtheorem{thm}{}
  {enhanced, theorem name and number, code={\edef\@currentlabelname{#2}}, 
  frame code={
        % \path[thick, draw] (frame.north west) -| (frame.north east) -| (frame.south east) -| (frame.south west) -| (frame.north west);
        \path[thick, draw] (frame.north west)  +(.5\baselineskip,0) -| +(0,-.5\baselineskip);
        % \path[thick, draw] (frame.north east) +(-.5\baselineskip,0) -| +(0,-.5\baselineskip);
        % \path[thick, draw] (frame.south west) +(.5\baselineskip,0) -| +(0,.5\baselineskip);
        \path[thick, draw] (frame.south east) +(-.5\baselineskip,0) -| +(0,.5\baselineskip);
    },
    left=1mm, right=1mm, top=1mm, bottom=1mm,
    colback=black!5,
    colframe=red!75!black,
    colbacktitle=black!0,
    coltitle=black!100,
    fonttitle=\bfseries}{thm}


\usepackage{environ}
\RenewEnviron{math}{%
\begin{align*}
\BODY
\end{align*}
}

\title{CS217 -- Algorithm Design and Analysis \\ Homework 5}
\date{\today}
\author{Not Strong Enough}


\begin{document}
\begin{thm}{}{}
    Describe an optimal solution of the dual program.
\end{thm}

\begin{proof}[Solution]
    Let $c(e)$ be lengths of edges and 
    $$d(u) := \begin{cases}
        0, u = s \\
        \text{the length of the shortest path from } s \text{ to } u, \text{ otherwise }
    \end{cases}$$

    Without loss of generality, we fix $z_s' = 0$. We claim that $z_u' = d(u)$ is an optimal solution. 
    \begin{itemize}
        \item $\forall v \in V$, $z_v' = d(v) \in \mathbf R $.
        \item $\forall e = (u, v) \in E$, assume that $d(v) - d(u) > c(e)$. This suggests that the length of the shortest path from $s$ to $t$ with length $d(u) + c(e)$, which is a contradiction to the definition.
        \item $z_t' - z_s'$ is maximized by letting $z_v' = d(v)$:
        
            Assume that $z_t' - z_s' = z_t' > d(t)$.

            Let the shortest path from $s$ to $t$ be $v_1, v_2, \cdots , v_r$ where $v_1 = s$ and $v_r = t$. We have $d(v_{i+1}) = d(v_i) + c((v_i, v_{i+1}))$.

            We next prove that $\forall i \in [r], z_{v_i}' > d({v_i})$ by induction from $r$ to $1$.
            \begin{itemize}
                \item Base: $z_{v_r}' > d(v_r)$ by our previous assumption.
                \item Inductive step: By inductive hypothesis, we have $z_{v_{i+1}}' > d(v_{i+1})$ for $i \in [r - 1]$.

                    Then $z_{v_i}' + c(e_i) \ge z_{v_{i+1}}' > d(v_{i+1}) = d(v_i) + c(e_i) \implies z_{v_i}' > d(v_i)$.
            \end{itemize}

            Hence, $z_s' > d(s) = 0$, which is a contradiction.
    \end{itemize}

    Therefore, $z_u' = d(u)$ is an optimal solution to the dual program.
\end{proof}
\end{document}
